\documentclass[professionalfonts,unicode]{beamer}

\usepackage{amsmath,amssymb}
\usepackage[utf8]{inputenc}

\usepackage[russian]{babel}

\usepackage{ifthen}

\usetheme{Warsaw}
\usecolortheme{uranix}

\setbeamertemplate{headline}
{%
  \begin{beamercolorbox}[sep=0.3cm,wd=\paperwidth]{section in head/foot}%
    \usebeamerfont{frametitle}%
    \vbox{}\vskip-1ex%
    \strut\insertsectionhead\strut\par%
    \vskip-1ex%
  \end{beamercolorbox}%
}
\setbeamertemplate{navigation symbols}{}
\setbeamertemplate{footline}{}
\setbeamertemplate{caption}[numbered]

\graphicspath{{images//}}

\title[Нелиненые уравнения]{Подготовка к контрольной. Задача Коши}
\author[Цыбулин И.В.]{Скалько Юрий Иванович\\
\textbf{Цыбулин Иван}
\\Шевченко Александр}
\date{}
%\vspace{0.3cm}

\newcommand\mypar{\medskip\newline}

\newcommand\myframe[3][dup]{
\ifthenelse{\equal{#1}{}}{}{\ifthenelse{\equal{#1}{dup}}{\subsection{#2}}{\subsection{#1}}}
\frame{\frametitle{#2}{#3}}%
}

\begin{document}

{
\setbeamertemplate{headline}[default]
\frame{\titlepage}
}

\section{Задача Коши}
\myframe{Задача}
{
	Для задачи Коши
	\begin{align*}
	y'(t) &= f(t,y(t))\\
	y(0) &= y_0
	\end{align*}
	Используется следующая численная схема
	\begin{align*}
	&u_{n+1} - u_{n-1} = \frac{h}{3}\left(f(t_{n-1},u_{n-1})+4f(t_n,u_n)+f(t_{n+1},u_{n+1})\right)\\
	&u_0 = y_0
	\end{align*}
	\begin{enumerate}
		\item Описать способ нахождения $u_2, u_3, \dots$
		\item Задать $u_1$ так, чтобы метод имел максимальный порядок аппроксимации (какой?)
		\item Показать устойчивость метода \emph {по начальным данным}. Чему равна константа устойчивости?
	\end{enumerate}
}

\myframe[Нахождение u\{n\}]{Нахождение $u_n$}
{
	Равенство
	$$u_{n+1} - u_{n-1} = \frac{h}{3}\left(f(t_{n-1},u_{n-1})+4f(t_n,u_n)+f(t_{n+1},u_{n+1})\right)$$
	задает при известных $u_{n},u_{n-1}$ \emph{уравнение} на $u_{n+1}$.
	\begin{align*}
	F(u_{n+1}) &= u_{n+1} - \frac{h}{3}f(t_{n+1},u_{n+1}) + \\
	&+\left[u_{n-1}+\frac{h}{3}\left(f(t_{n-1},u_{n-1})+4f(t_n,u_n)\right)\right] = 0
	\end{align*}
	Данное уравнение обычно решается методом Ньютона с начальным приближением $u_{n+1}^{(0)} = u_n$
	$$
	u_{n+1}^{(s+1)} = u_{n+1}^{(s)} - \left[F'(u_{n+1}^{(s)})\right]^{-1} F(u_{n+1}^{(s)}) 
	$$
}

\myframe[Задание u1]{Задание $u_1$}
{
	Чтобы иметь возможность вычислить хотя бы $u_2$ необходимо знать $u_0$ и $u_1$. 
	Требуется задать $u_1$ каким-то образом.
	\mypar\pause
	Ошибка аппроксимации всей задачи будет равна самой большой из ошибок аппроксимации
	уравнения и ошибки аппроксимации начального условия $u_1$. Поэтому, нет смысла аппроксимировать
	$u_1$ с большим или меньшим порядком, по сравнению с порядком аппроксимации уравнения.
	Найдем порядок аппроксимации уравнения $y'(t) = f(t,y(t))$ разностной схемой
	$$
	u_{n+1} - u_{n-1} = \frac{h}{3}\left(f(t_{n-1},u_{n-1})+4f(t_n,u_n)+f(t_{n+1},u_{n+1})\right)
	$$
}

\myframe{Ошибка аппроксимации уравнения}
{
	Подставим в качестве $u_n = [y]_n$, где $y(t)$ "--- решение
	Из-за симметрии разностной схемы относительно точки $n$, разложения в ряд Тейлора запишем именно относительно $t_n$.
	$$
	[y]_{n \pm 1} = [y]_n \pm h[y']_n + \frac{h^2}{2}[y'']_n \pm \frac{h^3}{6}[y''']_n + \frac{h^4}{24}[y^{IV}]_n + O(h^5)
	$$
	$$
	f(t_{n\pm 1}, [y]_{n\pm 1}) = [y']_{n \pm 1} = [y']_n \pm h[y'']_n + \frac{h^2}{2}[y''']_n \pm \frac{h^3}{6}[y^{IV}]_n + O(h^4)
	$$
	Раскладывая левую часть $[y]_{n+1}-[y]_{n-1}$, получим
	$$
	[y]_{n+1}-[y]_{n-1} = 2h\left([y']_n + \frac{h^2}{6}[y''']_n + O(h^4)\right)
	$$
	Разложение правой имеет вид
	$$
	\frac{h}{3}\left([f]_{n-1}+4[f]_n+[f]_{n+1}\right) = \frac{h}{3}\left(6[y']_n+h^2[y''']_n + O(h^4)\right)
	$$
}

\myframe[]{Ошибка аппроксимации уравнения}
{
	Разностное уравнение аппроксимирует дифференциальное уравнение $y = f(t, y(t))$ в точке $t_n$, умноженное на $2h$.
	Из-за этого множителя $2h$ схема не является устойчивой по определению. ``Константа'' устойчивости в этом случае зависит от $h$ 
	$$C = O(h)$$
	Легко построить устойчивую разностную схему (с $C = O(1)$), достаточно просто разделить схему на $O(h)$.
	$$\frac{u_{n+1} - u_{n-1}}{2h} = \frac{1}{6}\left(f(t_{n-1},u_{n-1})+4f(t_n,u_n)+f(t_{n+1},u_{n+1})\right)$$
	Теперь хорошо видно, что левая часть приближает $y'(t)$, а правая $f(t,y(t))$. Константа устойчивости 
	(при условии, что схема действительно устойчива), будет равна $O(1)$. Ошибку аппроксимации следует вычислять именно для 
	такого уравнения
}

\myframe[]{Ошибка аппроксимации уравнения}
{
	Вернемся к разложениям левой и правой частей разностного уравнения
	$$
	[y]_{n+1}-[y]_{n-1} = 2h\left([y']_n + \frac{h^2}{6}[y''']_n + O(h^4)\right)
	$$
	$$
	\frac{h}{3}\left([f]_{n-1}+4[f]_n+[f]_{n+1}\right) = \frac{h}{3}\left(6[y']_n+h^2[y''']_n + O(h^4)\right)
	$$
	После деления на $2h$ видно, что левая и правая части отличаются на величину порядка $O(h^4)$, это 
	различие и будет ошибкой аппроксимации.
	\mypar\pause
	Уравнения аппроксимированы с четвертым порядком. Значит необходимо задать $u_1$ также с четвертым порядком, т.е.
	$$
	u_1 - y(h) = O(h^4)
	$$
}

\myframe[Задание u\{1\}]{Задание $u_1$}
{
	Разложим $y(h)$ в ряд Тейлора вплоть до $O(h^4)$
	$$
	y(h) = y(0) + hy'(0) + \frac{h^2}{2} y''(0) + \frac{h^3}{6}y'''(0) + O(h^4)
	$$
	Воспользуемся
	\begin{align*}
		y(0) &= y_0, \quad y'(0) = f(0,y_0), \quad y''(0) = f_t(0,y_0) + f_y(0,y_0)f(0,y_0)\\
		y'''(0) &= f_{tt}(0,y_0) + 2f_{ty}(0,y_0)f(0,y_0)+f_{yy}(0,y_0)f^2(0,y_0) +\\
		&+f_y(0,y_0)f_t(0,y_0) + f_y^2(0,y_0)f(0,y_0)
	\end{align*}
	Тогда
	$$
	u_1 = [y]_0 + h[f]_0 + \frac{h^2}{2}\left([f_t]_0+[f_y]_0[f]_0\right) + 
	$$
	$$
	+ \frac{h^3}{6}\left(
	[f_{tt}]_0 + 2[f_{ty}]_0[f]_0 + [f_{yy}]_0[f]_0^2 + [f_{y}]_0[f_t]_0 + [f_{y}]_0^2[f]_0
	\right)
	$$
}

\myframe[]{Задание $u_1$}
{
	Задать $u_1$ можно и не зная аналитического выражения для $f(t,y)$. Для этого можно просто воспользоваться некоторым методом 
	четвертого порядка, для которого достаточно только одного начального условия $u_0$. 
	\mypar\pause
	Например, можно воспользоваться следующим методом (метод Рунге-Кутты 4го порядка)
	\begin{align*}
	k_1 &= f(0,y_0)\\
	k_2 &= f\left(\frac{h}{2},y_0+\frac{h}{2}k_1\right)\\
	k_3 &= f\left(\frac{h}{2},y_0+\frac{h}{2}k_2\right)\\
	k_4 &= f\left(h,y_0+hk_3\right)\\
	u_1 &= y_0 + \frac{h}{6}(k_1+2k_2+2k_3+k_4) = y(h) + O(h^4)
	\end{align*}
}

\myframe{Устойчивость по начальным данным}
{
	Пусть $u_n$ "--- решение разностной задачи с начальными условиями $u_0 = y_0, u_1 = y_1$. 
	Рассмотрим возмущенную задачу для $v_n$, которая отличается от исходной
	только начальными условиями $v_0 = y_0 + \delta_0, v_1 = y_1 + \delta_1$. Если 
	$$
	\| u_n - v_n \| < C \max(\delta_0, \delta_1),
	$$
	причем $C$ не зависит от $h$, то задача называется устойчивой по начальным данным.
	\mypar\pause
	Данное определение отличается от обычного определения устойчивости тем, что возмущение допускается только в 
	начальных условиях, когда для обычно устойчивости оно допускается и в правой части.
}

\myframe[]{Устойчивость по начальным данным}
{
	Для удобства введем $\Delta_n = |u_n - v_n|$. Тогда
	$$
	\Delta_0 = \delta_0, \qquad \Delta_1 = \delta_1
	$$
	\pause
	Докажем устойчивость в предположении, что $f(t,y)$ Липшицева по $y$, то есть
	$$
	|f(t,y) - f(t,g)| < L |y-g|
	$$
	\pause
	Запишем возмущенное и невозмущенное уравнения
	$$u_{n+1} = u_{n-1} + \frac{h}{3}\left(f(t_{n-1},u_{n-1})+4f(t_n,u_n)+f(t_{n+1},u_{n+1})\right)$$
	$$v_{n+1} = v_{n-1} + \frac{h}{3}\left(f(t_{n-1},v_{n-1})+4f(t_n,v_n)+f(t_{n+1},v_{n+1})\right)$$
	Для $\Delta_{n+1}$ справедливо
	$$
	\Delta_{n+1} \leq \Delta_{n-1} + \frac{Lh}{3}\Delta_{n-1} + \frac{4Lh}{3}\Delta_n + \frac{Lh}{3}\Delta_{n+1}
	$$
}

\myframe[]{Устойчивость по начальным данным}
{
	$$
	\Delta_{n+1} \leq \Delta_{n-1} + \frac{Lh}{3}\Delta_{n-1} + \frac{4Lh}{3}\Delta_n + \frac{Lh}{3}\Delta_{n+1}
	$$
	$$
	\left(1-\frac{Lh}{3}\right)\Delta_{n+1} \leq \left(1 + \frac{Lh}{3}\right)\Delta_{n-1} + \frac{4Lh}{3}\Delta_n
	$$
	Пусть $h$ мало настолько, что $A = Lh < 3$. Тогда
	$$
	\Delta_{n+1} \leq \frac{3+A}{3-A} \Delta_{n-1}+ \frac{4A}{3-A} \Delta_{n-1} \leq \frac{3+5A}{3-A} \max(\Delta_{n-1}, \Delta_{n}) \leq
	$$
	$$
	\leq \left(\frac{3+5A}{3-A}\right)^2 \max(\Delta_{n-2}, \Delta_{n-1}) \leq \dots 
	\leq \left(\frac{3+5A}{3-A}\right)^{n-1} \max(\Delta_{1}, \Delta_{0})
	$$
	Получаем
	$$
	\|u_n - v_n\| = \max \Delta_n \leq \left(\frac{3+5A}{3-A}\right)^{N-1} \max(\delta_{1}, \delta_{0})
	$$
}

\myframe[]{Устойчивость по начальным данным}
{
	$$
	\|u_n - v_n\| = \max \Delta_n \leq \left(\frac{3+5A}{3-A}\right)^{N-1} \max(\delta_{1}, \delta_{0})
	$$
	Покажем, что $\left(\frac{3+5A}{3-A}\right)^{N-1}$ можно ограничить константой, не зависящей от $h$.
	$$
	\left(\frac{3+5A}{3-A}\right)^{N-1} =	\exp\left\{(N-1)\ln\left(\frac{3+5Lh}{3-Lh}\right)\right\} = 
	$$
	$$
	= \exp\left\{(N-1)\ln\left(1+\frac{6Lh}{3}+O(h^2)\right)\right\} = e^{2LNh+O(h)} \rightarrow e^{2LT}
	$$
	В пределе при $h \rightarrow 0$ выражение $\left(\frac{3+5A}{3-A}\right)^{N-1}$ стремится к $e^{2LT}$,
	значит его можно ограничить константой не зависящей от $h$. Эта константа не будет сильно отличаться от
	$e^{2LT}$, поэтому можно принять, что константа устйочивости равна $e^{2LT}$
}

%%%%%%%%%%%%%%%%%%%%%%%%%%%%%%%%%%%%%%%%%%%%%%%
%%%%%%%%%%%%%%%%%%%%%%%%%%%%%%%%%%%%%%%%%%%%%%%
%%%%%%%%%                            %%%%%%%%%%
%%%%%%%%%%%%%%%%%%%%%%%%%%%%%%%%%%%%%%%%%%%%%%%
%%%%%%%%%%%%%%%%%%%%%%%%%%%%%%%%%%%%%%%%%%%%%%%
{
\setbeamertemplate{headline}[default] 
\frame{
	\begin{center}
	{\Huge Спасибо за внимание!}
	\end{center}
	\bigskip
	\begin{center}
	{\color{blue}{tsybulinhome@gmail.com}}
	\end{center}
	}
}

\end{document}
