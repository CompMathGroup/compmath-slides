\documentclass[12pt]{article}
\usepackage[utf8]{inputenc}
\usepackage[T2A]{fontenc}
\usepackage[english,russian]{babel}
\usepackage{amssymb,amsmath}

\renewcommand{\vec}[1]{\boldsymbol{\mathbf{#1}}}
\newcommand{\tr}{\mathsf{T}}

\author{Цыбулин Иван}
\title{Методы Рунге-Кутты \\ \vspace{2 mm} {\large Условия порядка. Функции устойчивости}}
\date{}

\begin{document}
	
\maketitle

\textbf{T5}. \emph{Рассматривается следующее однопараметрическое семейство однократно диагонально-неявных методов Рунге-Кутты:
\[
\begin{array}{c|cc}
	\gamma & \gamma & 0\\
	1-\gamma & 1-2\gamma & \gamma\\
	\hline
	& 1/2 & 1/2
\end{array}
\]
Найти все значения параметра $\gamma$, при которых:
\begin{itemize}
	\item[а)] метод является правильным (интерполяционным)
	\item[б)] метод имеет третий порядок аппроксимации,
	\item[в)] метод является асимптотически устойчивым (L-устойчивым),
	\item[г)] метод является A-устойчивым.
\end{itemize}
}

а) Метод Рунге-Кутты называется правильным или интерполяционным, если все коэффициенты в его таблице Бутчера неотрицательны. Данный метод будет правильным, если
\[
	0 \leqslant \gamma \leqslant \frac{1}{2}
\]
Методы, не являющиеся правильными склонны накапливать и усиливать ошибки округления.

б) Условия порядка методов Рунге-Кутты при выполнении условий Кутты
\[
c_i = \sum_{j=1}^s a_{ij}
\]
имеют вид 
\begin{itemize}
	\item для первого порядка и выше: $\displaystyle\sum_i b_i = 1$
	\item для второго порядка и выше: $\displaystyle\sum_i b_i c_i = \frac{1}{2}$
	\item для третьего порядка и выше: 
	\begin{gather*}
		\displaystyle\sum_i b_i c_i^2 = \frac{1}{3}\\
		\displaystyle\sum_{i,j} b_i a_{ij} c_j = \frac{1}{6}
	\end{gather*}
\end{itemize}
Считаем, что коэффициенты $a_{ij},b_i,c_j$ расположены в таблице Бутчера следующим образом
\[
\begin{array}{c|c}
\vec{c} & \vec{A}\\
\hline
\rule{0em}{2.5ex} & \vec{b}^\tr
\end{array}
\]

Данный метод удовлетворяет условиям Кутты
\begin{gather*}
	c_1 = \gamma = \gamma + 0 = \sum_j a_{1j}\\
	c_2 = 1 - \gamma = 1 - 2\gamma + \gamma = \sum_j a_{2j},
\end{gather*}
а также условиям первого
\begin{gather*}
	\sum_i b_i = \frac{1}{2} + \frac{1}{2} = 1
\end{gather*}
и второго порядка
\begin{gather*}
\sum_i b_i c_i = \frac{\gamma}{2} + \frac{1-\gamma}{2} = \frac{1}{2}
\end{gather*}
при любых $\gamma$. Выясним, когда метод дополнительно удовлетворяет условиям третьего порядка
\begin{gather*}
	\sum_{i} b_i c_i^2 = \frac{1}{2}\left(\gamma^2 + (1-\gamma)^2\right) 
	= \gamma^2 - \gamma + \frac{1}{2} = \frac{1}{3}\\
	\sum_{ij} b_i a_{ij} c_j = \frac{1}{2}\left(\gamma^2 + (1-2\gamma)\gamma + \gamma(1-\gamma)\right) = \gamma - \gamma^2 = \frac{1}{6}	
\end{gather*}
Оба условия свелись к одному квадратному уравнению
\[
\gamma^2 - \gamma + \frac{1}{6} = 0, \qquad \gamma_{1,2} = \frac{3 \pm \sqrt{3}}{6}.
\]
Корень $\gamma = \frac{3-\sqrt{3}}{6} \approx 0.211$ соответствует правильному (интерполяционному) методу третьего порядка, корень $\gamma = \frac{3+\sqrt{3}}{6} \approx 0.789$ --- другому методу третьего порядка, но с отрицательными коэффициентами.

в) Для исследования поведения метода на жестких задачах, применим его к модельному уравнению 
\[
y' = \lambda y,\quad\lambda \in \mathbb{C}
\]
При этом, полученный разностный метод можно записать в виде
\[
u_{n+1} = r(\tau \lambda) u_n,
\]
где $r(z)$ --- функция устойчивости метода, которая для методов Рунге-Кутты может быть записана как
\[
r(z) = \frac{\operatorname{det}(\vec E - z \vec A + z\vec B)}{\operatorname{det}(\vec E - z\vec A)}.
\]
В последнем выражении $\vec E$ --- единичная матрица, а $\vec B$ --- матрица, в которой коэффициенты $\vec b$ повторяются в каждой строке
\[
	\vec B = \begin{Vmatrix}
	b_1 & b_2 & \dots & b_s\\
	b_1 & b_2 & \dots & b_s\\
	&&\vdots\\
	b_1 & b_2 & \dots & b_s
	\end{Vmatrix}
\]
Для данного метода 
\begin{gather*}
r(z) = \frac{\begin{vmatrix}
	1 - \gamma z + z/2 & z/2\\
	1 + 2\gamma z - z/2 & 1 - \gamma z + z/2
	\end{vmatrix}}{\begin{vmatrix}
	1 - \gamma z & 0\\
	1 + 2\gamma z - z & 1 - \gamma z
	\end{vmatrix}}
	= \frac{P(z)}{Q(z)}\\
P(z) = 1 + (1 - 2\gamma) z + (1 - 4\gamma + 2\gamma^2)\frac{z^2}{2}\\
Q(z) = (1 - \gamma z)^2 = 1 - 2\gamma z + \gamma^2 z^2.
\end{gather*}
Метод будет $L$-устойчивым тогда, когда $r(z) \rightarrow 0$ при $z \rightarrow -\infty$.
\[
\lim_{z \rightarrow -\infty} r(z) = \lim_{z \rightarrow -\infty} \frac{P(z)}{Q(z)} = 
\frac{1 - 4 \gamma + 2 \gamma ^2}{\gamma^2}
\]
$L$-устойчивость возможна лишь при
\[
1 - 4\gamma + 2 \gamma^2 = 0, \quad \gamma_{1,2} = \frac{2 \pm \sqrt{2}}{2}
\]
Из этих двух $L$-устойчивых методов правильным будет метод с $\gamma = \frac{2 - \sqrt{2}}{2} \approx 0.293$.

г) Чтобы исследовать метод на $A$-устойчивость, рассмотрим область устойчивости метода
\[z \in \mathbb{C} : |r(z)| \leqslant 1.\]
Для $r(z) = \frac{P(z)}{Q(z)}$ можно дать эквивалентное определение.
\[z \in \mathbb{C} : |P(z)| \leqslant |Q(z)|\]

Метод будет $A$-устойчив, если $|r(z)| \leqslant 1$ для всей левой полуплоскости $\operatorname{Re} z \leqslant 0$. Рассмотрим область $\operatorname{Re} z \leqslant 0$. Ее границей будет мнимая ось $\operatorname{Re} z = 0$. Для этой области существует принцип максимума (принцип Фрагмена — Линделёфа), который гласит, что
функция, которая растет медленнее, чем $e^{|z|}$, и аналитична в области $\operatorname{Re} z \leqslant 0$ достигает своего максимума на границе, то есть на мнимой оси. 

Функция $r(z) = \frac{P(z)}{Q(z)}$ является отношением многочленов, поэтому не может экспоненциально расти. Дополнительно, единственный полюс $z = \frac{1}{\gamma}$ функции $r(z)$ расположен вне левой полуплоскости, а значит, функция $r(z)$ аналитическая в левой полуплоскости.

Рассмотрим мнимую ось и функцию $r(z) = r(iy)$ на ней. Всего возможно два варианта: \begin{itemize}
	\item $\max\limits_{y \in \mathbb{R}} |r(iy)| \leqslant 1$. Из принципа максимума, во всей области $\operatorname{Re} z \leqslant 0$ функция устойчивости по модулю не превышает 1, и метод является $A$-устойчивым
	\item $\max\limits_{y \in \mathbb{R}} |r(iy)| > 1$. Метод не может быть $A$-устойчивым, так как некоторые точки (вместе с некоторой окрестностью) на мнимой оси не принадлежат области устойчивости.
\end{itemize}

Вместо проверки $|r(iy)| \leqslant 1$ проще проверять $|P(iy)|^2 \leqslant |Q(iy)|^2$. Пусть $E(y) \equiv |Q(iy)|^2-|P(iy)|^2$. Для нашего метода
\begin{multline*}
E(y) = |1 -2\gamma iy - \gamma^2 y^2|^2 - \left|1 + (1-2\gamma)iy-(1-4\gamma+2\gamma^2) \frac{y^2}{2}\right|^2 = \\ =
(1 - \gamma^2 y^2)^2 + 4 \gamma^2 y^2 - \left(1 -(1-4\gamma+2\gamma^2) \frac{y^2}{2}\right)^2 - (1 - 2 \gamma)^2 y^2 = \\ = 1 - 1 + (-2\gamma^2 + 4\gamma^2 - 1 + 4\gamma - 4 \gamma^2 + 1 -4\gamma + 2\gamma^2)y^2 + \\ +
\gamma^4 y^4 - \frac{(1 - 4\gamma + 2\gamma^2)^2}{4}y^4 = \\ =
\frac{y^4}{4}(4\gamma - 1)(1-2\gamma)^2
\end{multline*}
Функция $E(y)$ неотрицательна при $\gamma \geqslant \frac{1}{4}$, а значит метод будет $A$-устойчивым при $\gamma \geqslant \frac{1}{4}$


\end{document}
