\documentclass[professionalfonts,compress,unicode]{beamer}

\usepackage{amsmath,amssymb}
\usepackage[utf8]{inputenc}
\usepackage[russian]{babel}

\usepackage{ifthen}

\usetheme{Dolgoprudny}
\usecolortheme{FUPM}

\renewcommand{\thefootnote}{\fnsymbol{footnote}}

\graphicspath{{images//}}

\title[МНК]{Переопределенные системы линейных уравнений}
\author[Цыбулин И.В.]{Скалько Юрий Иванович\\
\textbf{Цыбулин Иван}
\\Шевченко Александр}
\date{}

\begin{document}

{
	\setbeamertemplate{headline}{}
	\setbeamertemplate{footline}{}
	\frame{\titlepage}
}

\newcommand\myframe[3][dup]{
\ifthenelse{\equal{#1}{}}{}{
\ifthenelse{\equal{#1}{dup}}
{\subsection{#2}}{\subsection{#1}}
}\frame{\frametitle{#2}{#3}}}

\section{ }

\def\A{\mathbf{A}}
\def\B{\mathbf{B}}
\def\C{\mathbf{C}}
\def\D{\mathbf{D}}
\def\E{\mathbf{E}}
\def\L{\mathbf{L}}
\def\U{\mathbf{U}}
\def\W{\mathbf{W}}
\def\S{\mathbf{S}}
\def\J{\mathbf{J}}
\def\x{\mathbf{x}}
\def\y{\mathbf{y}}
\def\v{\mathbf{v}}
\def\w{\mathbf{w}}
\def\f{\mathbf{f}}
\def\b{\mathbf{b}}
\def\r{\mathbf{r}}
\def\p{\mathbf{p}}
\def\z{\mathbf{z}}

\section{Переопределенные СЛАУ}
\myframe{Переопределенные СЛАУ}
{
	Рассмотрим СЛАУ
	$$
	\A \x = \b,
	$$
	но в случае, когда уравнений больше, чем неизвестных. В этом случае система называется переопределенной.
	Почти всегда переопределенная система не имеет точного решения.
}

\myframe{Решение переопределенной СЛАУ}
{
	Поскольку точного решения такая система почти никогда не имеет, необходимо ввести другое понятие решения 
	переопределенной СЛАУ. Введем, как и раньше, невязку $\r = \A\x - \b$. Для точного решения требовалось $\r = 0$.
	Назовем решением переопределенной системы такое значение $\x$ для которого невязка минимальна
	$$
	\|\r\|_E^2 \rightarrow \min_{\x}
	$$
	
	Такое решение называется решением по методу наименьших квадратов
}

\myframe{Метод наименьших квадратов}
{
	Запишем квадрат нормы невязки
	$$
	\|\r\|_E^2 = \r^T\r = (\A\x-\b)^T(\A\x-\b) = \x^T\A^T\A\x - 2\A^T\b + \b^T\b
	$$
	Минимум достигается в точке, где градиент этой функции обращается в ноль, то есть
	$$
	\nabla\|\r\|_E^2 = 2(\A^T\A\x-\A^T\b) = 0
	$$
	Рассмотрим систему 
	$$
	\A^T\A \x = \A^T \b
	$$
	Это система с квадратной матрицей, которая не вырождена, если ранг матрицы $\A$ равен числу неизвестных.
	Решение этой системы будет решением по методу наименьших квадратов.
}

\section{Поиск оптимальных параметров}
\myframe{Метод наименьших квадратов}
{
	Метод широко применяется при анализе экспериментальных данных для определения параметров зависимостей, проведения наилучших прямых и т.п.
	Пусть имеется набор данных $(x_i, y_i), i=\overline{1,n}$ и требуется по ним построить наилучшую прямую. Запишем уравнение прямой 
	$$ y = \alpha x + \beta$$
	Здесь $\alpha, \beta$ - неизвестные, причем на эти неизвестные имеется $n$ уравнений
	$$ y_i = \alpha x_i + \beta, i = \overline{1,n}$$
}

\myframe[]{Метод наименьших квадратов}
{
	$$ y_i = \alpha x_i + \beta, i = \overline{1,n}$$
	Запишем эту систему в матричной форме
	$$
	\begin{pmatrix}
		x_1&1\\
		x_2&1\\
		\vdots&1\\
		x_n&1
	\end{pmatrix}
	\begin{pmatrix}
		\alpha\\\beta
	\end{pmatrix} = 
	\begin{pmatrix}
		y_1\\y_2\\\vdots\\y_n
	\end{pmatrix}
	$$
	Решая ее по методу наименьших квадратов, необходимо решить систему
	$$
	\begin{pmatrix}
		x_1&x_2&\dots&x_n\\
		1&1&1&1
	\end{pmatrix}	
	\begin{pmatrix}
		x_1&1\\
		x_2&1\\
		\vdots&1\\
		x_n&1
	\end{pmatrix}
	\begin{pmatrix}
		\alpha\\\beta
	\end{pmatrix} = 
	\begin{pmatrix}
		x_1&x_2&\dots&x_n\\
		1&1&1&1
	\end{pmatrix}		
	\begin{pmatrix}
		y_1\\y_2\\\vdots\\y_n
	\end{pmatrix}
	$$	
}
\myframe[]{Метод наименьших квадратов}
{
	$$
	\begin{pmatrix}
		x_1&x_2&\dots&x_n\\
		1&1&1&1
	\end{pmatrix}	
	\begin{pmatrix}
		x_1&1\\
		x_2&1\\
		\vdots&1\\
		x_n&1
	\end{pmatrix}
	\begin{pmatrix}
		\alpha\\\beta
	\end{pmatrix} = 
	\begin{pmatrix}
		x_1&x_2&\dots&x_n\\
		1&1&1&1
	\end{pmatrix}		
	\begin{pmatrix}
		y_1\\y_2\\\vdots\\y_n
	\end{pmatrix}
	$$	
	Перемножая матрицы, получим
	$$
	\begin{pmatrix}
		\sum x_i^2&\sum x_i\\
		\sum x_i&n\\
	\end{pmatrix}
	\begin{pmatrix}
		\alpha\\\beta
	\end{pmatrix} =
	\begin{pmatrix}
		\sum x_i y_i\\
		\sum y_i
	\end{pmatrix}
	$$
	\begin{align*}
		\alpha &= \frac{n \sum x_i y_i - \sum x_i \sum y_i}
		{n \sum x_i^2 - \left(\sum x_i\right)^2}\\
		\beta &= \frac{\sum x_i^2 \sum y_i - \sum x_i \sum x_i y_i}
		{n \sum x_i^2 - \left(\sum x_i\right)^2}\\
	\end{align*}
}

\myframe{Более сложные зависимости}
{
	Методом наименьших квадратов можно искать наилучшие коэффициенты 
	и в нелинейных зависимостях. Важно, чтобы искомые коэффициенты входили
	в зависимость линейно.

	Например, рассмотрим зависимость
	$$
	y = \alpha + \beta x + \gamma x^3
	$$

	Сама зависимость нелинейна, но коэффициенты $\alpha, \beta, \gamma$ входят 
	в нее линейно.
}

\myframe[]{Более сложные зависимости}
{
	Аналогично
	$$ y_i = \alpha + \beta x_i + \gamma x_i^3, i = \overline{1,n}$$
	Запишем эту систему в матричной форме
	$$
	\begin{pmatrix}
		1&x_1&x_1^3\\
		1&x_2&x_2^3\\
		1&\vdots&\vdots\\
		1&x_n&x_n^3\\
	\end{pmatrix}
	\begin{pmatrix}
		\alpha\\\beta\\\gamma
	\end{pmatrix} = 
	\begin{pmatrix}
		y_1\\y_2\\\vdots\\y_n
	\end{pmatrix}
	$$
	Решая ее по методу наименьших квадратов, необходимо решить систему
	$$
	\begin{pmatrix}
		1&1&\dots&1\\
		x_1&x_2&\dots&x_n\\
		x_1^3&x_2^3&\dots&x_n^3
	\end{pmatrix}
	\begin{pmatrix}
		1&x_1&x_1^3\\
		1&x_2&x_2^3\\
		1&\vdots&\vdots\\
		1&x_n&x_n^3\\
	\end{pmatrix}
	\begin{pmatrix}
		\alpha\\\beta\\\gamma
	\end{pmatrix} = 
	\begin{pmatrix}
		1&1&\dots&1\\
		x_1&x_2&\dots&x_n\\
		x_1^3&x_2^3&\dots&x_n^3
	\end{pmatrix}
	\begin{pmatrix}
		y_1\\y_2\\\vdots\\y_n
	\end{pmatrix}
	$$	
}

\myframe[]{Более сложные зависимости}
{
	$$
	\begin{pmatrix}
		1&1&\dots&1\\
		x_1&x_2&\dots&x_n\\
		x_1^3&x_2^3&\dots&x_n^3
	\end{pmatrix}
	\begin{pmatrix}
		1&x_1&x_1^3\\
		1&x_2&x_2^3\\
		1&\vdots&\vdots\\
		1&x_n&x_n^3\\
	\end{pmatrix}
	\begin{pmatrix}
		\alpha\\\beta\\\gamma
	\end{pmatrix} = 
	\begin{pmatrix}
		1&1&\dots&1\\
		x_1&x_2&\dots&x_n\\
		x_1^3&x_2^3&\dots&x_n^3
	\end{pmatrix}
	\begin{pmatrix}
		y_1\\y_2\\\vdots\\y_n
	\end{pmatrix}
	$$
	Умножая матрицы
	$$
	\begin{pmatrix}
		n & \sum x_i & \sum x_i^3\\
		\sum x_i & \sum x_i^2 & \sum x_i^4\\
		\sum x_i^3 & \sum x_i^4 & \sum x_i^6
	\end{pmatrix}
	\begin{pmatrix}
		\alpha\\\beta\\\gamma
	\end{pmatrix} = 
	\begin{pmatrix}
		\sum y_i\\
		\sum x_i y_i\\
		\sum x_i^3 y_i
	\end{pmatrix}
	$$
	Решение последней системы дает наилучшие значения $\alpha, \beta, \gamma$.
}

\myframe[]{Нелинейные зависимости от искомых коэффициентов}
{
	В некоторых случаях неизвестные коэффициенты входят в зависимости нелинейно.
	Иногда можно изменить уравнение так, чтобы коэффициенты входили линейно
	\begin{align*}
	\sin (\omega x_i + \varphi) = y_i &\rightarrow 
		\omega x_i + \varphi = \arcsin y_i\\
	\rho_i = c T_i^\gamma &\rightarrow
		\ln \rho_i = \ln c + \gamma \ln T_i
	\end{align*}

	Во втором случае неопределенными коэффициентам будут $\xi \equiv \ln c$ и $\gamma$.
	После нахождения $\xi$ можно легко найти $c = e^\xi$
}

\myframe{Взвешенный метод наименьших квадратов}
{
	Допустим, по некоторым причинам, некоторым уравнениям в системе приписан больший вес в суммарной невязке. Это означает, что 
	одни уравнения должны выполняться точнее других, или наоборот, некоторые уравнения вообще не должны влиять на результат.

	Пусть дана диагональная матрица $\W = \W^T > 0$. Вместо минимизации $\|\r\|_E^2 = \r^T\r$ будем 
	минимизировать $\|\r\|_W^2 \equiv \r^T\W\r$. 

	Чем больше значение $W_{ii}$ тем точнее должно выполняться уравнение $i$
	(его невязка учитывается с большим весом)
	$$
	\r^T \W \r = (\x^T\A^T - \b^T)\W(\A\x - \b) = \x^T\A^T\W\A\x - 2\b^T \W\A\x +\b^T\b
	$$
	Аналогично, получаем обычную СЛАУ
	$$
	\A^T\W\A \x = \A^T\W\b
	$$
}

\myframe[]{Взвешенный метод наименьших квадратов}
{
	Пусть имеются те же самые данные $(x_i,y_i)$, но еще дополнительно известна погрешность $\delta y_i$ 
	(например, погрешность измерений). Имеет смысл вместо задачи
	$$
	\sum_i (y_i - \alpha x_i - \beta)^2 \rightarrow \min_{\alpha, \beta}
	$$
	решать задачу
	$$
	\sum_i \left(\frac{y_i - \alpha x_i - \beta}{\delta y_i}\right)^2 \rightarrow \min_{\alpha, \beta}
	$$
	Тогда матрица $\W = diag\left(\frac{1}{\delta y_1^2},\dots,\frac{1}{\delta
	y_n^2}\right)$.
	Чем больше ошибка измерения, тем меньше <<вес>> уравнения.
}
%%%%%%%%%%%%%%%%%%%%%%%%%%%%%%%%%%%%%%%%%%%%%%%
%%%%%%%%%%%%%%%%%%%%%%%%%%%%%%%%%%%%%%%%%%%%%%%
%%%%%%%%%                            %%%%%%%%%%
%%%%%%%%%%%%%%%%%%%%%%%%%%%%%%%%%%%%%%%%%%%%%%%
%%%%%%%%%%%%%%%%%%%%%%%%%%%%%%%%%%%%%%%%%%%%%%%
{
	\setbeamertemplate{headline}{}
	\setbeamertemplate{footline}{}
	\frame{
		\begin{center}
		{\Huge Спасибо за внимание!}
		\end{center}
		\bigskip
		\begin{center}
		{\color{blue}{tsybulinhome@gmail.com}}
		\end{center}
	}
}

\end{document}
