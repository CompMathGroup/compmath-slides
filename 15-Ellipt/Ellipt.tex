\documentclass[professionalfonts,compress,unicode]{beamer}

\usepackage{amsmath,amssymb}
\usepackage{empheq}
\usepackage[utf8]{inputenc}

\usepackage[russian]{babel}

\usepackage{ifthen}

\def\[#1\]{\begin{align*}#1\end{align*}}

\newcommand\myframe[3][dup]{
\ifthenelse{\equal{#1}{}}{}{\ifthenelse{\equal{#1}{dup}}{\subsection{#2}}{\subsection{#1}}}
\frame{\frametitle{#2}{#3}}%
}

\usetheme{Warsaw}
\usecolortheme{uranix}

\setbeamertemplate{headline}
{%
  \begin{beamercolorbox}[sep=0.3cm,wd=\paperwidth]{section in head/foot}%
    \usebeamerfont{frametitle}%
    \vbox{}\vskip-1ex%
    \strut\insertsectionhead\strut\par%
    \vskip-1ex%
  \end{beamercolorbox}%
}
\setbeamertemplate{navigation symbols}{}
\setbeamertemplate{footline}{}
\setbeamertemplate{caption}[numbered]

\renewcommand{\thefootnote}{\fnsymbol{footnote}}

\graphicspath{{images//}}

\title[Уравнение Пуассона]{Уравнения эллиптического типа.\\ Уравнение Пуассона}
\author[Цыбулин И.В.]{Скалько Юрий Иванович\\
\textbf{Цыбулин Иван}}
\date{}
%\vspace{0.3cm}

\newcommand{\cutefrac}[2]{{}^{#1}\!/{\!}_#2}
\newcommand{\half}{\cutefrac{1}{2}}

\begin{document}

{
\setbeamertemplate{headline}[default]
\frame{\titlepage}
}

\section{Задача Дирихле для уравнения Пуассона}

\myframe{Уравнение Пуассона}
{
	Чаще всего на практике уравнения эллиптического типа представлены уравнением
	Пуассона
	\[
		\frac{\partial ^2 u}{\partial x^2} +
		\frac{\partial ^2 u}{\partial y^2}
		\equiv \Delta u = -\rho
	\]

	Обычно это уравнение решается в некоторой области, и для корректной
	постановки задачи требуется задать какие-то граничные условия. Задача
	Дирихле предполагает, что на границе известны значения функции $u$.

	Мы будем решать задачу Дирихле для уравнения Пуассона в квадратной области
	$G = [0, 1] \times [0, 1]$ с нулевыми граничными условиями $u\big|_{\partial
	G} = 0$
}

\myframe{Схема на шаблоне крест}
{
	Введем в квадрате $[0, 1] \times [0, 1]$ равномерную сетку с шагом $h =
	\frac{1}{N}$ по каждому направлению.

	Заменим дифференциальный оператор Лапласа его конечно-разностным
	приближением
	\[
	\Delta \rightarrow \Lambda \equiv \Lambda_{xx} + \Lambda_{yy}.
	\]
	Здесь $\Lambda_{xx}, \Lambda_{yy}$ --- разностные операторы вторых
	производных по $x$ и $y$.
	\[
	\Lambda_{xx} u_{n,m} = \frac{u_{m+1,n} - 2 u_{m,n} + u_{m-1,n}}{h^2}\\
	\Lambda_{yy} u_{n,m} = \frac{u_{m,n+1} - 2 u_{m,n} + u_{m,n-1}}{h^2}
	\]
	При этом получается разностное уравнение
	\[
	\frac{u_{m+1,n} + u_{m,n+1} - 4 u_{m,n} + u_{m-1,n}	+ u_{m,n-1}}{h^2} =
	-\rho_{m,n}
	\]
}

\myframe[]{Схема на шаблоне крест}
{
	Разностное уравнение необходимо дополнить граничными условиями:
	\begin{gather*}
	\frac{u_{m+1,n} + u_{m,n+1} - 4 u_{m,n} + u_{m-1,n}	+ u_{m,n-1}}{h^2} =
	-\rho_{m,n}, \quad 0 < m,n < N\\
	u_{0,n} = u_{N,n} = u_{m,0} = u_{m, N} = 0
	\end{gather*}

	Данная задача является системой линейных уравнений относительно неизвестных 
	$u_{m,n}$. Каждому внутреннему сеточному узлу соответствует ровно одна неизвестная
	и ровно одно уравнение.
}

\myframe{Линейные операторы и матрицы}
{
	Хотя каждый линейный оператор (например, разностный оператор Лапласа) в
	конечномерном пространстве может быть представлен в виде матрицы, это не
	всегда удобно.

	Например, оператор $\Lambda$ действует по простому
	\emph{правилу}, то есть легко вычислить как он подействует на конкретную
	сеточную функцию $u_{n,m}$. С другой стороны, хранить его в виде матрицы
	$(N-1)^2 \times (N-1)^2$ крайне неэффективно.

	Некоторые методы для решения систем линейных уравнений работают только с
	матрицами (например, метод Гаусса или, что то же самое, метод LU
	разложения). Напротив, многие итерационные методы используют только
	вычисления умножения матрицы на вектор (или другими словами, действие
	оператора на сеточную функцию).
}

\section{Методы решения разностного уравнения Пуассона}
\myframe{Метод простой итерации с параметром}
{
	Учтем, что оператор $-\Lambda$ положительно определен и
	симметричен. Рассмотрим метод простой итерации с параметром $\tau$ для
	уравнения $-\Lambda u_{m,n}= \rho_{m,n}$:
	\[
	u_{m,n}^{(p+1)} = u_{m,n}^{(p)} + \tau \left(\Lambda u_{m,n}^{(p)} +
	\rho_{m,n}\right)
	\]
	Здесь $p$ --- номер очередного приближения к решению. Начальное приближение
	можно задать нулевое $u_{m,n}^{(0)} = 0$.

	Оптимальным выбором $\tau$ будет значение
	\[
	\tau = \frac{2}{\lambda_{\min}(-\Lambda) + \lambda_{\max}(-\Lambda)}
	\]
	Найдем это оптимальное значение.
}

\myframe{Оптимальное значение итерационного параметра}
{
	Для каждого из одномерных операторов верно
	\[
	\lambda_k(\Lambda_{xx}) = -\frac{4}{h^2}\sin^2 \frac{\pi k}{2 N}, \quad k =
	\overline{1, N-1}
	\]
	Тогда 
	\begin{gather*}
	\lambda_{\min}(-\Lambda) = \frac{8}{h^2} \sin^2 \frac{\pi}{2N}\\
	\lambda_{\max}(-\Lambda) 
		= \frac{8}{h^2} \sin^2 \frac{(N-1)\pi}{2N}
		= \frac{8}{h^2} \cos^2 \frac{\pi}{2N}\\
	\end{gather*}
	Получается следующее значение оптимального параметра
	\[
	\tau_{\text{opt}} = \frac{2h^2}{8} = \frac{h^2}{4}
	\]
}

\myframe{Метод Якоби}
{
	Изучим более подробно разностную схему с оптимальным параметром.
	Оказывается, ее можно записать в форме метода Якоби:
	\begin{gather*}
	u_{m+1,n}^{(p)} + u_{m,n+1}^{(p)} - 4 u_{m,n}^{(p+1)} + u_{m-1,n}^{(p)}	+
	u_{m,n-1}^{(p)} = -h^2\rho_{m,n}\\
	u_{0,n}^{(p)} = u_{N,n}^{(p)} = u_{m,0}^{(p)} = u_{m, N}^{(p)} = 0
	\end{gather*}

	На каждом шаге в каждом внутреннем узле решение пересчитывается по значениям
	соседей на предыдущем шаге.
}

\myframe{Метод Зейделя}
{
	Метод Зейделя можно получить расставляя над неизвестными в соседних узлах
	номер итерации: $(p)$ или $(p+1)$, в зависимости от того, в каком порядке
	обходятся неизвестные. Метод Зейделя решает уравнения
	\emph{последовательно}, друг за другом.
	\begin{gather*}
	u_{m+1,n}^{(p)} + u_{m,n+1}^{(p)} - 4 u_{m,n}^{(p+1)} + u_{m-1,n}^{(p+1)} +
	u_{m,n-1}^{(p+1)} = -h^2\rho_{m,n}\\
	u_{0,n}^{(p)} = u_{N,n}^{(p)} = u_{m,0}^{(p)} = u_{m, N}^{(p)} = 0
	\end{gather*}
	Здесь предполагается, что к моменту решения уравнения для $(m,n)$ уравнения
	для $(m-1,n)$ и $(m,n-1)$ были уже решены на итерации $p+1$.
}

\myframe{Метод установления}
{
	Снова вернемся к схеме простой итерации с параметром и перепишем ее в виде
	\[
	\frac{u_{m,n}^{(p+1)} - u_{m,n}^{(p)}}{\tau} = \Lambda u_{m,n}^{(p)} +
	\rho_{m,n}
	\]
	Это --- простейшая явная схема для уравнения теплопроводности, где $p$
	теперь играет роль номера шага по <<времени>>.

	Решение уравнения Пуассона является пределом при $t \rightarrow \infty$
	решения уравнения теплопроводности
	\[
	\frac{\partial u}{\partial t} = \Delta u + \rho
	\]
	Методы, основанные на этом факте называются методами установления.
}

\myframe{Схема расщепления для решения уравнения Пуассона}
{
	Рассмотрим схему расщепления для уравнения теплопроводности
	\[
	\frac{w^{(p+1/2)}_{m,n} - u^{(p)}_{m,n}}{\tau/2} = \Lambda_{xx} w^{(p+1/2)}_{m,n} +
	\Lambda_{yy} u^{(p)}_{m,n} + \rho_{m,n}\\
	\frac{u^{(p+1)}_{m,n} - w^{(p+1/2)}_{m,n}}{\tau/2} = \Lambda_{xx}
	w^{(p+1/2)}_{m,n} + \Lambda_{yy} u^{(p+1)}_{m,n} + \rho_{m,n}
	\]
	Она безусловно устойчива, то есть сходится к решению уравнения Пуассона при
	любом $\tau$. Однако, оптимальный выбор $\tau$ может значительно ускорить
	сходимость.
}

\myframe{Метод быстрого преобразования Фурье}
{
	Оказывается, что для уравнения Пуассона существует \emph{прямой}
	метод решения, который допускает эффективную реализацию. 

	Поскольку для оператора $-\Lambda$ известны все собственные функции и
	собственные значения, решение уравнения Пуассона можно выписать явно.

	\[
	-\Lambda \psi_{m,n}^{(k,l)} = \lambda_{k,l} \psi_{m,n}^{(k,l)},
	\]
	где
	\begin{gather*}
	\lambda_{k,l} = \frac{4}{h^2}\left(\sin^2\frac{\pi k}{2N} + \sin^2 \frac{\pi
	l}{2N}\right)\\
	\psi_{m,n}^{k,l} = \sin \frac{\pi k m}{2N} \sin \frac{\pi l n}{2N}
	\end{gather*}\
}

\myframe[]{Метод быстрого преобразования Фурье}
{
	Собственные функции оператора $-\Lambda$ ортогональны относительно
	скалярного произведения
	\[
	\left(u_{m,n}, v_{m,n}\right) = \sum_{m,n=1}^{N-1} u_{m,n}v_{m,n}
	\]
	Решение уравнения Пуассона при этом записывается в виде
	\[
	u_{m,n} = \sum_{k,l} 
	\frac{
		\left(\rho_{m,n},\psi_{m,n}^{k,l}\right)
		}{
		\left(\psi_{m,n}^{(k,l)},\psi_{m,n}^{(k,l)}\right)
		}
		\frac{\psi_{m,n}^{(k,l)}}{\lambda_{k,l}}
	\]
	Скалярные произведения можно быстро вычислить с помощью дискретного
	преобразования Фурье.
}

%%%%%%%%%%%%%%%%%%%%%%%%%%%%%%%%%%%%%%%%%%%%%%%
%%%%%%%%%%%%%%%%%%%%%%%%%%%%%%%%%%%%%%%%%%%%%%%
%%%%%%%%%                            %%%%%%%%%%
%%%%%%%%%%%%%%%%%%%%%%%%%%%%%%%%%%%%%%%%%%%%%%%
%%%%%%%%%%%%%%%%%%%%%%%%%%%%%%%%%%%%%%%%%%%%%%%
{
\setbeamertemplate{headline}[default] 
\frame{
	\begin{center}
	{\Huge Спасибо за внимание!}
	\end{center}
	\bigskip
	\begin{center}
	{\color{blue}{tsybulin@crec.mipt.ru}}
	\end{center}
	}
}

\end{document}
