\documentclass[professionalfonts,compress,unicode]{beamer}

\usepackage{amsmath,amssymb}
\usepackage[utf8]{inputenc}

\usepackage[russian]{babel}

\usepackage{multirow}
\usepackage{colortbl}

\usetheme{Warsaw}
\usecolortheme{uranix}

\setbeamertemplate{headline}
{%
  \begin{beamercolorbox}[sep=0.3cm,wd=\paperwidth]{section in head/foot}%
    \usebeamerfont{frametitle}%
    \vbox{}\vskip-1ex%
    \strut\insertsectionhead\strut\par%
    \vskip-1ex%
  \end{beamercolorbox}%
}
\setbeamertemplate{navigation symbols}{}
\setbeamertemplate{footline}{}

\renewcommand{\thefootnote}{\fnsymbol{footnote}}

\graphicspath{{images//}}

\title[Интерполяция. Сплайны]{Интерполяция\\Сплайны}
\author[Цыбулин И.В.]{Скалько Юрий Иванович\\
\textbf{Цыбулин Иван}}
\date{}
%\vspace{0.3cm}

\begin{document}

{
\setbeamertemplate{headline}[default]
\frame{
\titlepage
}

%\frame{
%\frametitle{Содержание}
%\small
%%\tiny
%\tableofcontents
%}
}

\newcommand\myframe[2]{\subsection{#1}\frame{\frametitle{#1}{#2}}}

%\section{ }
%
%\myframe{Материалы по курсу вычислительной математики}
%{
%\begin{itemize}
%	\item 
%		Материалы курса (методички, лекции, учебники и др.) можно найти 
%		на сайте кафедры вычислительной математики
%		{\color{blue} http://crec.mipt.ru/study/materials/compmath/}
%	\item 
%		Любые вопросы по курсу (и не только) можно присылать на почтовый ящик
%		{\color{blue} tsybulinhome@gmail.com}
%\end{itemize}
%}

\section{Интерполяция}
\myframe{Задача интерполяции}
{
	\begin{block}{Задача}
	Предположим, что некоторая функция $f(x)$ известна в точках $\left\{ x_k \right\}_{k=1}^n$: $f(x_k) = f_k$. 
	Как определить ее значение в какой-нибудь другой точке $x^* \neq x_k$?
	\end{block}
	\pause
	
	Конечно, без дополнительных условий, данная задача некорректна. Функция может вести себя 
	в промежутках между заданными точками произвольно. Но оказывается, что при определенных условиях,
	исходную функцию можно достаточно хорошо \emph{приблизить} функцией из некоторого семейства так,
	чтобы она проходила через заданные точки $(x_k, f_k)$. Эта функция называется \emph{интерполянтом}
}

\myframe{Определения}
{
	Понятия ``узел'', ``сетка'', ``шаг сетки'' встречаются в вычислительной математике очень часто. 
	В отношении задачи интерполяции, \emph{узлами} называются точки $x_k$, то есть точки, в которых заданы значения функции. 
	\emph{Сеткой} называется совокупность всех узлов. \emph{Шагом сетки} называется расстояние между соседними узлами.
	Шаг может быть постоянным (равномерная сетка) или переменным (неравномерная сетка).
}

\myframe{Виды интерполяции}
{
	В зависимости от вида семейства функций интерполяция бывает
	\begin{itemize}
		\item алгебраической --- интерполянт является многочленом от $x$
		\pause
		\item {\color<3->{gray}тригонометрической --- интерполянт является тригонометрическим многочленом $Q_m(x) = $
		$$
			a_0 + a_1 \cos \frac{2\pi x}{L} + b_1 \sin \frac{2\pi x}{L} + \dots + a_m \cos \frac{2\pi mx}{L} + b_m \sin \frac{2\pi mx}{L}
		$$}
		\pause
		\item сплайновой --- интерполянт является кусочно-многочленной функцией. На каждом отрезке $[x_k, x_{k+1}]$ сплайн является многочленом, а
		в узлах ставятся дополнительные условия (непрерывность, гладкость и т.п.)
	\end{itemize}
}

\myframe{Алгебраическая интерполяция. СЛАУ}
{
	Будем искать многочлен $P(x) = a_0 + a_1 x + a_2 x^2 + \dots$, который удовлетворяет всем 
	равенствам $P(x_k) = f_k$. Неизвестными здесь будут коэффициенты многочлена $a_j$. 
	\pause
	\begin{equation*}
	\left\{
	\begin{array}{ccc}
	a_0 + a_1 x_1 + a_2 x_1^2 + \dots &=& f_1\\
	a_0 + a_1 x_2 + a_2 x_2^2 + \dots &=& f_2\\
	&\vdots&\\
	a_0 + a_1 x_n + a_2 x_n^2 + \dots &=& f_n
	\end{array}
	\right.
	\end{equation*}
	
	\pause
	\begin{equation*}
	\left(
	\begin{array}{ccccc}
		1&x_1&x_1^2& \dots & x_1^{n-1}\\
		1&x_2&x_2^2& \dots & x_2^{n-1}\\
		&&&\vdots\\
		1&x_n&x_n^2& \dots & x_n^{n-1}
	\end{array}
	\right)
	\left(
	\begin{array}{@{}c@{}}
		a_0\\
		a_1\\
		\vdots\\
		a_{n-1}
	\end{array}
	\right) = 
	\left(
	\begin{array}{@{}c@{}}
		f_1\\
		f_2\\
		\vdots\\
		f_n
	\end{array}
	\right)	
	\end{equation*}
}

\myframe{Алгебраическая интерполяция}
{
	Задача алгебраической интерполяции, таким образом, свелась
	к решению системы линейных алгебраических уравнений с матрицей 
	$$
	W = \left(
	\begin{array}{ccccc}
		1&x_1&x_1^2& \dots & x_1^{n-1}\\
		1&x_2&x_2^2& \dots & x_2^{n-1}\\
		&&&\vdots\\
		1&x_n&x_n^2& \dots & x_n^{n-1}
	\end{array}
	\right)
	$$
	\begin{block}{Вопрос}
		Как называется эта матрица? Чему равен ее определитель?
		\pause
		
		Эта матрица называется матрицей Вандермонда и ее определитель $\det W ={\prod}_{i<j}  (x_i-x_j) \neq 0$ при $x_i \neq x_j$
	\end{block}
	Получается, что задача алгебраической интерполяции всегда имеет решение, и при этом единственное --- многочлен степени $n-1$. 
}

\myframe{Алгебраическая интерполяция. Другие методы}
{
	Коэффициенты многочлена-интерполянта можно искать решая СЛАУ. Однако, существуют более
	простые и надежные методы построения этого многочлена, а именно
	\begin{itemize}
		\item Интерполяционный многочлен в форме Ньютона
		\item Интерполяционный многочлен в форме Лагранжа
	\end{itemize}
	\pause
	
	Необходимо понимать, что интерполянт остается все тем же единственным многочленом степени $n-1$, 
	проходящим через все точки $(x_k, f_k)$. Отличие заключается лишь в способе его построения.
	\pause
	
	Интерполяционный многочлен Ньютона проще строить на практике, но интерполяционный многочлен Лагранжа
	оказывается весьма удобным для теоретического изучения свойств интерполянтов.
}

\myframe{Интерполяционный многочлен в форме Ньютона}
{
	Построение интерполянта в форме Ньютона происходит путем последовательного добавления точек и 
	соответствующего ``подправления'' интерполянта.
	\begin{itemize}
	\pause
	\item Изначально есть только одно значение $f(x_1) = f_1$ и интерполянт просто равен константе $P(x) = f_1$.
	\pause
	\item Предположим, что интерполянт для первых $k$ точек уже посторен. Добавляем точку $(x_{k+1}, f_{k+1})$. 
	Чтобы не нарушить интерполяционное свойство, к интерполянту нужно добавить функцию, которая в точках $x_1 \div x_k$ обращается в ноль.
	Общий вид этой функции $A(x-x_1)(x-x_2)\dots(x-x_k)$. Значение $A$ оперделяется из требования $P(x_{k+1}) = f_{k+1}$
	\end{itemize}
}

\myframe{Пример интерполянта в форме Ньютона}
{
	\begin{block}{Построим интерполянт по следующим данным}
	$$
	\begin{array}{|c||c|c|c|}
	\hline
	x_k&1&2&4\\
	\hline
	f_k&1&3&1\\
	\hline
	\end{array}
	$$
	\end{block}
	\pause
	\begin{columns}[T]
		\begin{column}{0.45\textwidth}
			\begin{itemize}
			\item<2-> Полагаем $P(x)=1$.
			\item<3-> Добавляем линейную функцию к $P$: 
			$$P = 1 + A(x-1)$$
			\item<4-> Добавляем квадратичную функцию к $P$: 
			$$P = 1 + 2(x-1) + B(x-1)(x-2)$$
			\end{itemize}
		\end{column}
		\begin{column}{0.55\textwidth}
			\only<2>{\includegraphics[width=\textwidth]{p0.png}}
			\only<3>{\includegraphics[width=\textwidth]{p1.png}}
			\only<4>{\includegraphics[width=\textwidth]{p2.png}}
		\end{column}		
	\end{columns}
}

\myframe{Разделенные разности}
{
	Ньютон нашел выражения для неизвестных коэффициентов $A$ в форме, удобной 
	для вычислений. Для этого вводится понятие \emph{разделенной разности}.
	Разделенная разность $k$-го порядка обозначается как $f(\underbrace{x_p,x_q,\dots,x_s}_{k+1 \text{ аргумент}})$.
	Разделенные разности нулевого порядка совпадают со значениями самой функции в этой точке
	$$
	f(x_k) = f_k
	$$
	Остальные разности определяются рекуррентно:
	$$
	f(x_p,x_q,\dots,x_r,x_s) = \frac{f(x_q,\dots,x_r,x_s)-f(x_p,x_q,\dots,x_r)}{x_s - x_p}
	$$
	В этих обозначениях,
	$$
	P(x) = f(x_1) + f(x_1, x_2) (x-x_1) + f(x_1,x_2,x_3)(x-x_1)(x-x_2) + \dots
	$$
}

\myframe{Пример вычисления разделенных разностей}
{
	Вычисление разделенных разностей удобно проводить в виде таблицы:
	\pause
	\begin{table}
	\begin{tabular}{|c||c|c|c|c|c|c|}
	\hline
	$x_k$&
		\multicolumn{2}{p{1cm}}{\centering $\color<3,5>{blue}1$}&
		\multicolumn{2}{|p{1cm}}{\centering $\color<3,4>{blue}2$}&
		\multicolumn{2}{|p{1cm}|}{\centering $\color<4,5>{blue}4$}\\ \hline
	$f(x_k)$&
		\multicolumn{2}{p{1cm}}{\centering $\color<6>{blue}\color<3>{green}1$}&
		\multicolumn{2}{|p{1cm}}{\centering $\color<3,4>{green}3$}&
		\multicolumn{2}{|p{1cm}|}{\centering $\color<4>{green}1$}\\ \hline
	$f(x_k,x_{k+1})$&
		\multicolumn{1}{>{\columncolor[gray]{.5}}p{0.285cm}}{}&
		\multicolumn{2}{|p{1cm}}{\centering $\color<1,2>{white}\color<5,6>{green}\color<3>{red}2$}&
		\multicolumn{2}{|p{1cm}}{\centering $\color<1-3>{white}\color<5>{green}\color<4>{red}-1$}&
		\multicolumn{1}{|>{\columncolor[gray]{.5}}p{0.285cm}|}{}\\ \hline
	$f(x_k,x_{k+1},x_{k+2})$&
		\multicolumn{2}{>{\columncolor[gray]{.5}}p{1cm}}{}&
		\multicolumn{2}{|p{1cm}}{\centering $\color<1-4>{white}\color<5,6>{red}-1$}&
		\multicolumn{2}{|>{\columncolor[gray]{.5}}p{1cm}|}{}\\ \hline
	\end{tabular}
	\end{table}
	\only<1-2>{
	$$
	f(x_p,x_q,\dots,x_r,x_s) = \frac{f(x_q,\dots,x_r,x_s)-f(x_p,x_q,\dots,x_r)}{x_s - x_p}
	$$
	}
	\only<3>{
	$$
	{\color{red}f(x_1,x_2)} = \frac{{\color{green}f(x_2)}-{\color{green}f(x_1)}}{{\color{blue}x_2} - {\color{blue}x_1}}
	$$
	}
	\only<4>{
	$$
	{\color{red}f(x_2,x_3)} = \frac{{\color{green}f(x_3)}-{\color{green}f(x_2)}}{{\color{blue}x_3} - {\color{blue}x_2}}
	$$
	}	
	\only<5>{
	$$
	{\color{red}f(x_1,x_2,x_3)} = \frac{{\color{green}f(x_2,x_3)}-{\color{green}f(x_1,x_2)}}{{\color{blue}x_3} - {\color{blue}x_1}}
	$$
	}	
	\only<6>{
	$$
	\phantom{\frac{\phantom{f(x)}}{\phantom{f(x)}}}
	P(x) = {\color{blue} 1} + {\color{green} 2}(x-x_1) {\color{red} -1}(x-x_1)(x-x_2) 
	\phantom{\frac{\phantom{f(x)}}{\phantom{f(x)}}}
	$$
	}	
}

\myframe{Базисные интерполяционные полиномы}
{
	Для построения интерполяционного многочлена в форме Лагранжа решается вспомогательная задача
	\begin{block}{Задача о базисном интерполяционном многочлене}
		Необходимо построить многочлен, который во всех точках $x_k$, кроме точки $x_j$ обращался в 0,
		а в точке $x_j$ был равен $1$
		$$
		\ell_j(x_k) = \left\{
		\begin{array}{lcl}
		0&,& k\neq j\\
		1&,& k = j\\
		\end{array}
		\right.
		$$
	\end{block}		
	\pause
	Поскольку степень этого многочлена $n-1$, а $x_k, k\neq j$ - его корни, то сам многочлен можно записать в форме
	$$
	\ell_j(x) = A(x-x_1)(x-x_2)\cdots(x-x_{j-1})(x-x_{j+1})\cdots(x-x_n)
	$$
	\pause
	Пользуясь условием $\ell_j(x_j) = 1$
	$$
	\ell_j(x) = \frac{(x-x_1)(x-x_2)\cdots(x-x_{j-1})(x-x_{j+1})\cdots(x-x_n)}{(x_j-x_1)(x_j-x_2)\cdots(x_j-x_{j-1})(x_j-x_{j+1})\cdots(x_j-x_n)}
	$$	
}

\myframe{Интерполяционный многочлен в форме Лагранжа}
{
	Используя базисные интерполяционные многочлены Лагранжа легко написать явное выражение для интерполянта в форме Лагранжа
	$$
	P(x) = \sum_{j=1}^n \ell_j(x) f_j
	$$
	\pause
	Действительно,
	$$
	P(x_k) = \sum_{j=1}^n \ell_j(x_k) f_j = \ell_k(x_k) f_k = f_k
	$$
	\pause
	Заметим, что базисные интерполяционные многочлены $\ell_j(x)$ зависят только от \emph{сетки}, а не от значений 
	функции в узлах. Если приходится решать несколько задач интерполяции на одной и той же сетке, 
	то форма Лагранжа может оказаться удобнее.
}

\myframe{\large Пример вычисления базисных интерполяционных многочленов}
{
	$$
	\begin{array}{|c||c|c|c|}
	\hline
	x_k&1&2&4\\
	\hline
	f_k&1&3&1\\
	\hline
	\end{array}
	$$
	\begin{columns}[T]
		\begin{column}{0.55\textwidth}
		\vspace{-0.5cm}
		$$
		\ell_1(x) = \frac{(x-2)(x-4)}{(1-2)(1-4)} = \frac{1}{3}(x-2)(x-4)
		$$
		$$
		\ell_2(x) = \frac{(x-1)(x-4)}{(2-1)(2-4)} = \frac{1}{2}(x-1)(4-x)
		$$
		$$
		\ell_3(x) = \frac{(x-1)(x-2)}{(4-1)(4-2)} = \frac{1}{6}(x-1)(x-2)
		$$
		\end{column}
		\begin{column}{0.45\textwidth}
		\includegraphics[width=\textwidth]{l.png}
		\end{column}
	\end{columns}
	\pause
	
	\begin{block}{Вопрос}
		Чему равна сумма $\ell_1(x) + \ell_2(x) + \ell_3(x)$ ?
		\pause
		
		$\ell_1(x) + \ell_2(x) + \ell_3(x) = 1$. 
		
		Подсказка: рассмотреть $f(x) = 1$ и ее интерполянт $P(x)$
		
	\end{block}
}

\myframe{\large Пример интерполяционного многочлена в форме Лагранжа}
{
\center
\includegraphics[height=0.75\textheight]{lp.png}
$$
P(x) = \frac{1}{3}(x-2)(x-4) + 3\frac{1}{2}(x-1)(4-x) + \frac{1}{6}(x-1)(x-2)
$$
}

\myframe{Погрешность алгебраической интерполяции}
{
	Логичный вопрос --- насколько восстановленная по значениям функция (интерполянт)
	близка к исходной? Она в точности с ней совпадает в точках $x_k$, но что можно сказать 
	про различия в промежутках?
	\pause
	
	\begin{block}{Теорема}
	Ошибка алгебраической интерполяции допускает оценку
	$$
	\left| f(x) - P(x) \right| \leq \frac{f^{(n)}(\xi)}{n!}|\omega(x)| \leq \frac{M_n}{n!} |\omega(x)|, \quad x,\xi,x_k \in [a,b],
	$$
	где $\omega(x) = (x-x_1)(x-x_2)\cdots(x-x_n)$
	\end{block}
	\pause
	
	часть ошибки $\frac{M_n}{n!}$ зависит только от вида функции, а вторая $\omega(x)$ --- только от расположения точек интерполяции.
}

\myframe{Ощибка интерполяции на равномерной сетке}
{
	Рассмотри равномерную сетку $x_k = a + \frac{k-1}{n-1} (b-a)$
	Оценим максимальное значение функции $\omega(x)$ на ней. 
	$$
	\max_{x \in [a,b]} \omega(x) \leq (n-1)! \left(\frac{b-a}{n-1}\right)^n \equiv (n-1)! h^n, 
	$$
	где через $h$ обозначен шаг сетки, то есть $\frac{b-a}{n-1}$
	\pause
	
	Отсюда, погрешность интерполяции, которая является ошибкой метода, равна
	$$
	\varepsilon_{\text{метод}} = \frac{M_n}{n!} \max_{x \in [a,b]}| \omega(x) |\leq \frac{M_n}{n} h^n
	$$
	Однако, в ошибке метода фигурирует максимум $n$-й производной, который может сильно расти при увеличении $n$.
}

\myframe{Интерполяция функции Рунге на равномерной сетке}
{
	\only<1>{\includegraphics[height=0.8\textheight]{runge5.png}}
	\only<2>{\includegraphics[height=0.8\textheight]{runge10.png}}
	\only<3>{\includegraphics[height=0.8\textheight]{runge15.png}}
	\only<4>{\includegraphics[height=0.8\textheight]{runge20.png}}
	\only<5>{\includegraphics[height=0.8\textheight]{runge25.png}}
}

\myframe{Оптимальный выбор узлов интерполяции}
{
	Посмотрим, насколько возможно уменьшить ошибку интерполяции, только за счет выбора узлов $x_k$. (Предполагаем, что можем узнать
	только $n$ значений функции, но в тех точках, которые нам интересны). 
	\pause
	
	Задача состоит в минимизации функции $\omega(x)$ за счет выбора $x_k$. Однако фраза ``минимизация функции'' требует конкретизации
	\pause
	
	Если искать минимум максимального отклонения $\omega(x)$, то такая задача была решена Чебышевым(1881)%
	\footnote{О функциях мало удаляющихся от нуля при некоторых величинах переменной/Чебышев П.Л. - Спб.,1881}
	
	$$
	\max_{x \in [a,b]} \left| (x-x_1)(x-x_2) \cdots (x-x_n) \right| \rightarrow \min_{x_k}
	$$
}

\myframe{Многочлены Чебышева}
{
	Многочленом Чебышева степени $n$ называется многочлен
	$$
	T_n(x) = \cos n \arccos x = 2^{n-1} x^n + \dots
	$$
	Он является многочленом, наименее уклоняющимся от нуля на отрезке $[-1,1]$ среди многочленов с тем же коэффициентом при старшей степени.
	Чтобы получить решение предыдущей задачи, необходимо этот многочлен отмасштабировать и перевести отрезок $[-1,1]$ в $[a,b]$.
	\pause
	
	$$
	\omega(x) = \tilde{T}_n(x) = \frac{(b-a)^n}{2^{2n-1}} \cos n \arccos \frac{2x-a-b}{b-a}
	$$
	
	\pause
	$$
	\max_{x\in[a,b]} = \frac{(b-a)^n}{2^{2n-1}} = \frac{h^n n^n}{2^{2n-1}} \approx \frac{h^n n! e^n}{2^{2n-1}\sqrt{2\pi n}} =
	h^n n! \left(\frac{e}{4}\right)^n \sqrt\frac{2}{\pi n}
	$$
	
	Существенное отличие от равномерной сетки в быстро убывающем сомножителе $\left(\frac{e}{4}\right)^n$
}

\myframe{Сетка из нулей многочлена Чебышева}
{
	Узлы сетки $x_k$ являются корнями $\omega(x)$. Оптимальной в смысле минимума ошибки интерполяции будет
	сетка из узлов $x_k$, которые являются корнями $\omega(x) = \tilde{T}_n(x)$.
	\pause
	\begin{columns}[T]
	\begin{column}{0.6\textwidth}
	$$
	\tilde{T}_n(x) = \frac{(b-a)^n}{2^{2n-1}} \cos n \arccos \frac{2x-a-b}{b-a}
	$$
	$$
	x_k = \frac{a+b}{2} + \frac{b-a}{2} \cos \left(\frac{2k-1}{2n}\pi\right)
	$$
	\end{column}
	\begin{column}{0.4\textwidth}
	\includegraphics[width=0.9\textwidth]{cheb.png}
	\end{column}
	\end{columns}
	\pause
	
	\begin{block}{Теорема}
		Если функция $f(x)$ имеет ограниченную производную на отрезке, то последовательность 
		интерполяционных многочленов $P_n(x)$ на такой сетке сходится равномерно к $f(x)$.
		$$
		P_n(x) \rightrightarrows f(x)
		$$
	\end{block}
}

\myframe{Экстраполяция}
{
	До сих пор, мы изучали поведение интерполянта в пределах отрезка, на котором заданы точки.
	Также можно ставить задачу определения значений функции за пределами отрезка, например, 
	спрогнозировать значения функции по уже имеющимся данным.
	\pause
	
	Большая часть формальных выводов, в том числе и погрешности экстраполяции, 
	один-к-одному переносятся из интерполяции. Отличие заключается в расширении отрезка $[a,b]$,
	до отрезка, в который входит точка $x$. В свою очередь, оценки для максимумов функции $\omega(x)$
	сильно зависят от изучаемого отрезка.
}

\myframe{Экстраполяция на равномерной сетке}
{
	Для оценки ошибки экстраполяции остается верной формула
	$$
	\varepsilon_{\text{метод}} \leq \frac{M_n}{n!} |\omega(x)|
	$$
	Пусть точка $x$ лежит правее точки $b$ на $\delta$: $x = b + \delta$
	$$
	\omega(x) = \prod_{k=0}^{n-1} \left(\delta+kh\right) = h^n\frac{\Gamma\left(\frac{\delta}{h}+n\right)}{\Gamma\left(\frac{\delta}{h}\right)}
	\approx 
	\left\{
	\begin{array}{lcl}
		h^n n!&,& \delta \lesssim h\\
		\delta^n&,& \delta \gg h
	\end{array}
	\right.
	$$
	То есть, экстраполяция на расстояния порядка $h$ имеет погрешность, близкую к погрешности интерполяции,
	но по мере удаления от конца отрезка, ошибка стремительно растет.
}

\myframe{\large Экстаполяция на сетке из нулей многочлена Чебышева}
{
	В этом случае открывается другое экстремальное свойство многочленов Чебышева.
	\pause
	
	Наряду с тем, что на данной сетке функция $\omega(x)$ наименее отклоняется от нуля 
	среди всех многочленов со старшей степенью $1$, эта функция стремительнее всех остальных
	растет за пределами отрезка $[a,b]$. 
	\pause
	
	Таким образом, сетка из нулей многочлена Чебышева оказывается самой плохой
	в смысле погрешности экстраполяции --- оценка для ошибки превышает оценку для ошибки на любой другой сетке.
}

\myframe{Чувствительность интерполяции}
{
	Возьмем $20$ точек функции $\sin x$ и чуть-чуть (на доли процента) пошевелим значение функции в одной из них
	\begin{columns}
	\begin{column}[c]{0.75\textwidth}

	% comments MATTER!!!!! Keep 'em ----------------------------VVV
	\only<1>{\includegraphics[width=1.00\textwidth]{sense0.png}}%
	\only<2>{\includegraphics[width=1.00\textwidth]{sense2.png}}%
	\only<3>{\includegraphics[width=1.00\textwidth]{sense4.png}}%
	\only<4>{\includegraphics[width=1.00\textwidth]{sense6.png}}%
	\only<5>{\includegraphics[width=1.00\textwidth]{sense8.png}}%
	\only<6>{\includegraphics[width=1.00\textwidth]{sense10.png}}%
	\only<7>{\includegraphics[width=1.00\textwidth]{sense12.png}}%
	\only<8>{\includegraphics[width=1.00\textwidth]{sense14.png}}%
	\only<9>{\includegraphics[width=1.00\textwidth]{sense16.png}}%
	\only<10>{\includegraphics[width=1.00\textwidth]{sense18.png}}%
%	\only<11>{\includegraphics[width=1.00\textwidth]{sense10.png}}%
%	\only<12>{\includegraphics[width=1.00\textwidth]{sense11.png}}%
%	\only<13>{\includegraphics[width=1.00\textwidth]{sense12.png}}%
%	\only<14>{\includegraphics[width=1.00\textwidth]{sense13.png}}%
%	\only<15>{\includegraphics[width=1.00\textwidth]{sense14.png}}%
%	\only<16>{\includegraphics[width=1.00\textwidth]{sense15.png}}%
%	\only<17>{\includegraphics[width=1.00\textwidth]{sense16.png}}%
%	\only<18>{\includegraphics[width=1.00\textwidth]{sense17.png}}%
%	\only<19>{\includegraphics[width=1.00\textwidth]{sense18.png}}%
%	\only<20>{\includegraphics[width=1.00\textwidth]{sense19.png}}%

	\end{column}
	\begin{column}[c]{0.25\textwidth}
	
	\invisible<1>{\hyperlink{jumptofirst}{\beamergotobutton{К первому кадру}}}
	
	\invisible<10>{\hyperlink{jumptolast}{\beamergotobutton{К последнему кадру}}}

	\end{column}
	\end{columns}
	
	\hypertarget<1>{jumptofirst}{}
	\hypertarget<10>{jumptolast}{}
}

\myframe{Чувствительность интерполяции}
{
	Вспомним выражение для интерполяционного многочлена в форме Лагранжа
	$$
	P(x) = \sum_{j=1}^n f_j \ell_j(x)
	$$
	``Пошевелив'' $f_k$ на $\delta f_k$, мы тем самым ``пошевелили'' интерполянт на 
	$$
	\delta P(x) = \sum_{j=1}^n (f_j+\delta f_j) \ell_j(x) - \sum_{j=1}^n f_j \ell_j(x) = \sum_{j=1}^n \delta f_j \ell_j(x)
	$$
	Поскольку конкретное направление шевеления (в большую или меньшую сотрону) обычно неизвестно, а известно только
	абсолютное значение, можно написать оценку
	$$
	|\delta P(x)| \leq \sum_{j=1}^n |\delta f_j| |\ell_j(x)|
	$$
}

\myframe{Функция Лебега и константа Лебега}
{
	Рассмотрим случай, когда все $|\delta f_k|$ одинаковы и равны $\delta f$:
	$$
	|\delta P(x)| \leq \delta f \sum_{j=1}^n |\ell_j(x)|
	$$
	Сумма $\sum_{j=1}^n |\ell_j(x)|$ зависит только от сетки, называется \emph{функцией Лебега} этой сетки и обозначается $L(x)$.
	В случае, когда интересует максимальное отклонение интерполянта по всему отрезку, вводят максимум функции Лебега, 
	который называется \emph{константой Лебега} и обозначается $L$
	$$
	|\delta P(x)| \leq L(x) \delta f
	$$
	$$
	|\delta P| \leq \max_{x \in [a,b]} L(x)  \delta f \equiv L \delta f
	$$
}

\myframe{Функция Лебега равномерной сетки}
{
	\begin{figure}%
	\only<1>{\includegraphics[height=0.5\textheight]{leb_un5.png}}%
	\only<2>{\includegraphics[height=0.5\textheight]{leb_un10.png}}%
	\only<3->{\includegraphics[height=0.5\textheight]{leb_un20.png}}%
	\end{figure}
	
	Для равномерной сетки константа Лебега $L$ растет как $L \sim \frac{2^n}{\sqrt{n}}$.
	
	\uncover<4>{
	Также видно, что за пределами отрезка функция Лебега растет еще быстрее. 
	Это означает что задача экстраполяции крайне чувствительна
	к заданию точных значений в узлах.
	}
}

\myframe{\large Функция Лебега сетки из нулей многочлена Чебышева}
{
	\begin{figure}%
	\only<1>{\includegraphics[height=0.5\textheight]{leb_ch5.png}}%
	\only<2>{\includegraphics[height=0.5\textheight]{leb_ch10.png}}%
	\only<3->{\includegraphics[height=0.5\textheight]{leb_ch20.png}}%
	\end{figure}
	
	Для этой сетки константа Лебега $L$ растет как $L \sim \frac{2}{\pi}\ln n$.
	\pause
	
	Использование сетки из нулей многочлена Чебышева позволяет сильно снизить требования к 
	точности задания функции в узлах.
}

\section{Сплайны}
\myframe{Проблемы глобальной интероляции}
{
	Глобальная многочленная или тригонометрическая интерполяции при большом количестве
	узлов начинают испытывать проблемы при быстром росте констант $M_n$ и весьма
	чувствительны к заданию функции в узлах.
	
	\pause
	Одно из решений --- проводить не глобальную, а локальную интерполяцию, 
	по небольшому количеству соседних узлов. Такой интерполянт называется сплайном.
	
	\pause
	\emph{Степенью сплайна} называется степень многочлена на каждом отрезке. 
	\emph{Гладкостью сплайна} называется количество непрерывных производных у функции на \emph{всем} отрезке
	\emph{Дефектом сплайна} называется разность между степенью и гладкостью сплайна.
}

\myframe{Кусочно-линейная интерполяция}
{
	Простейшая кусочно-многочленная интерполяция --- кусочно линейная. 
	Функция на каждом отрезке приближается линейной.
	
	\begin{columns}[c]
	\begin{column}{0.6\textwidth}
	\begin{figure}
	\center
	\includegraphics[width=\textwidth]{spline1_1.png}%
	\end{figure}
	\end{column}
	\begin{column}{0.4\textwidth}
	Степень --- 1
	
	Гладкость --- 0 
	
	Дефект --- 1
	\end{column}
	\end{columns}
}

\myframe{Кусочно-квадратичная интерполяция}
{
	Построим на каждом отрезке параболу по трем ближайшим точкам.
	\begin{columns}[c]
	\begin{column}{0.6\textwidth}
	\begin{figure}
	\center
	\includegraphics[width=\textwidth]{spline2_2.png}%
	\end{figure}
	\end{column}
	\begin{column}{0.4\textwidth}
	Степень --- 2
	
	Гладкость --- 0 
	
	Дефект --- 2
	\end{column}
	\end{columns}
}

\myframe{Гладкая кусочно-квадратичная интерполяция}
{
	Построим по трем первым точкам параболу, а на следующих отрезках будем стоить параболу, проходящую
	через концы отрезка и гладко продолжающую параболу на предыдущем отрезке.
	
	Пусть $P_k(x) = a_kx^2+b_kx+c$, $q_k = P'_{k-1}(x_k)$
	$$
	\left\{
	\begin{array}{lcl}
		a_kx_k^2+b_kx_k+c_k &=& f_k\\
		a_kx_{k+1}^2+b_kx_{k+1}+c_k &=& f_{k+1}\\
		2a_kx_k+b_k &=& q_k\\
	\end{array}
	\right.
	$$
	%$$
	%\left\{
	%\begin{array}{lcl}
		%a_k &=&  \frac{f_{k+1}-f_k+q(x_{k}-x_{k+1})}{(x_{k+1}-x_k)^2}\\
		%b_k &=&  \frac{2x_k(f_{k}-f_{k+1})+q(x_{k+1}^2-x_k^2)}{(x_{k+1}-x_k)^2}\\
		%c_k &=&  \frac{f_{k+1}x_{k}^2+x_{k+1}\left(qx_k(x_k-x_{k+1})+f_k(x_{k+1}-2x_k)\right)}{(x_{k+1}-x_k)^2}\\
		%q_{k+1} &=&  \frac{f_{k+1}-f_k}{x_{k+1}-x_k} - q_k
	%\end{array}
	%\right.
	%$$
	Данный метод позволяет строить сплайны любой степени с дефектом 1. Частный случай степени 3 называется сплайном Шонберга.
}

\myframe{Гладкая кусочно-квадратичная интерполяция}
{
	
	\begin{columns}[c]
	\begin{column}{0.6\textwidth}
	\begin{figure}
	\center
	\includegraphics[width=\textwidth]{spline2_1.png}%
	\end{figure}
	\end{column}
	\begin{column}{0.4\textwidth}
	Степень --- 2
	
	Гладкость --- 1
	
	Дефект --- 1
	\end{column}
	\end{columns}
	
	Удалось добиться гладкости сплайна, но при этом исчезло свойство локальности:
	при изменении какого-нибудь значения функции изменяется весь сплайн. Конечно, изменение не такое 
	большое, как при глобальной интерполяции, но хотелось бы от него избавиться
}

\myframe{Локальные гладкие сплайны}
{
	Возьмем за основу негладкий сплайн $P(x)$(например кусочно-линейный). 
	На каждом отрезке будем искать кубическую параболу $Q_k(x)$, которая проходит через его концы,
	на левом конце производная совпадает с $P'(x_k+0)$, а на правом --- с $P'(x_{k+1}+0)$. 
	Таким образом, производная сплайна будет непрерывной, а для вычисления интерполянта на отрезке используются только 
	3 ближайшие точки
	$$
	\left\{
	\begin{array}{lcl}
		Q_k'(x_k) &=& \frac{f_{k}-f_{k-1}}{x_k-x_{k-1}}\\
		Q_k'(x_{k+1}) &=& \frac{f_{k+1}-f_{k}}{x_{k+1}-x_{k}}\\
		Q_k(x_k) &=& f_k\\
		Q_k(x_{k+1}) &=& f_{k+1}\\
	\end{array}
	\right.
	$$
	Таким образом можно строить локальные сплайны степени $2s+1$ при гладкости $s$.
}

\myframe{Гладкая локальная кусочно-кубическая интерполяция}
{
	
	\begin{columns}[c]
	\begin{column}{0.6\textwidth}
	\begin{figure}
	\center
	\includegraphics[width=\textwidth]{spline3_1.png}%
	\end{figure}
	\end{column}
	\begin{column}{0.4\textwidth}
	Степень --- 3
	
	Гладкость --- 1
	
	Дефект --- 2
	\end{column}
	\end{columns}

	Сплайн получился гладкий и сохранил свойство локальности. Такие локальные сплайны называются сплайнами В.С. Рябенького.
}
%%%%%%%%%%%%%%%%%%%%%%%%%%%%%%%%%%%%%%%%%%%%%%%
%%%%%%%%%%%%%%%%%%%%%%%%%%%%%%%%%%%%%%%%%%%%%%%
%%%%%%%%%                            %%%%%%%%%%
%%%%%%%%%%%%%%%%%%%%%%%%%%%%%%%%%%%%%%%%%%%%%%%
%%%%%%%%%%%%%%%%%%%%%%%%%%%%%%%%%%%%%%%%%%%%%%%
{
\setbeamertemplate{headline}[default] 
\frame{
	\begin{center}
	{\Huge Спасибо за внимание!}
	\end{center}
	\bigskip
	\begin{center}
	{\color{blue}{tsybulin@crec.mipt.ru}}
	\end{center}
	}
}

\end{document}
