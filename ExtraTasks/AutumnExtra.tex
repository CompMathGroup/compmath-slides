\documentclass[12pt]{article}
\usepackage[utf8]{inputenc}
\usepackage[T2A]{fontenc}
\usepackage[english,russian]{babel}
\usepackage{indentfirst,amssymb,amsmath}

\newcounter{task}

\newcommand{\problem}{\par\textbf{\stepcounter{task}\thetask. }}

\title{Дополнительные задачи по курсу вычислительной математики. 5 семестр}
\author{Цыбулин Иван}

\begin{document}
\maketitle

\problem Используя ваш любимый язык программирования, напишите функцию,
вычисляющую
функцию Бесселя первого рода $J_0(x)$, суммируя часть ее ряда Тейлора в
окрестности точки $x_0 = 0$:
\[
J_0(x) = \sum_{k = 0}^{\infty} \frac{(-1)^k}{(k!)^2}
\left(\frac{x}{2}\right)^{2k}.
\]

Далее, используя эту функцию и формулу численного дифференцирования,
найдите производную функции Бесселя $J_0(x)$ в точке $x = 1$ с заданной
точностью
$\varepsilon = 10^{-6}$.

Если в программе используются константы, такие как число членов ряда Тейлора или
значение шага дифференцирования, должно быть указано, как они получены.

\emph{Примечание}. Для дифференцирования использовать оптимальный шаг $h^*$.
Принять, что погрешность, с которой вычисляются значения $J_0(x)$ равна ошибке
метода вычисления функции с помощью отрезка ее ряда Тейлора, и может быть
принята равной первому отброшенному слагаемому в ряде Тейлора. Использовать
минимальное число членов ряда Тейлора для решения этой задачи. Для оценки
максимумов всех производных функции Бесселя использовать $M_n \leq 1$.

\problem Построить для функции $f(x) = \sin x$ на отрезке $[0, 2\pi]$ кубический
сплайн $P(x)$, который в узлах интерполяции $x_k$ совпадает со значениями $f(x_k)$,
производная которого непрерывна, и в узлах совпадает со значениями $f'(x_k) =
\cos x_k$ (такой сплайн называется сплайном Эрмита). Численно изучить зависимость
максимального отклонения $f(x)$ и $P(x)$ в зависимости от количества узлов $N$.
Найти порядок метода (степень зависимости максимального отклонения от
$h = \frac{2\pi}{N - 1}$)

\problem Написать программу, которая используя правило Рунге и формулу
Гаусса численного интегрирования четвертого порядка (ошибка на всем
отрезке имеет порядок $O(h^4)$)
\[
\int_{x_1}^{x_2} f(x) dx \approx \frac{x_2 - x_1}{2}\left[
f\left(\frac{x_1+x_2}{2} - \frac{x_2 - x_1}{2\sqrt{3}}\right) +
f\left(\frac{x_1+x_2}{2} + \frac{x_2 - x_1}{2\sqrt{3}}\right)
\right]
\]
вычисляет интеграл
\[
\int_0^{\pi/2} \ln \frac{1}{\sin x} dx
\]
с точностью $\varepsilon = 10^{-10}$. Сравнить полученное значение с точным
значением интеграла $\frac{\pi}{2} \ln 2$

\emph{Указание}. Убедиться, что $\Delta_h = |I_h - I_{h/2}| \sim O(h^4)$. Если это
не так, регуляризуйте подынтегральную функцию.

\problem Для решения интегрального уравнения Фредгольма первого рода
\[
\int_{-\pi}^\pi K(x-y) u(y) = f(x)
\]
с ядром $K(s) = |s|$ и правой частью $f(x) = (\pi^2 + x^2)\cos^2 \frac{x}{2}$
используется квадратурная формула средней точки. При этом интегральное уравнение
сводится к следующей системе линейных уравнений
\[
h \sum_{m=1}^{N} K_{nm} u_{m} = f_{n}, \quad n = 1,\dots,N
\]
где $K_{nm} = h|n-m|$ --- симметричная матрица ядра размера $N\times N$, а \mbox{$f_n =
f\left(nh-\frac{h}{2}\right)$}. Точное решение интегрального уравнения $U(x) =
\cos^2 \frac{\pi}{2}$, численно должно совпадать с ним с точностью до небольшой
ошибки метода порядка $O(h^2)$.

Решить полученную систему одним из следующих методов и сравнить решение с точным
$U_m = U\left(mh-\frac{h}{2}\right)$:
\begin{itemize}
\item[\textbullet${}^{\phantom{*}}$] с помощью $LU$-разложения;
\item[\textbullet${}^{*}$] с помощью $LUP$-разложения;
\item[\textbullet${}^{*}$] с помощью $LL^\mathsf{T}$-разложения (Холецкого);
\item[\textbullet${}^{*}$] с помощью $QR$-разложения методом вращений Гивенса;
\item[\textbullet${}^{*}$] с помощью $QR$-разложения методом отражений
Хаусхолдера.
\end{itemize}
Возмутить правую часть системы случайным вектором $\delta f_n$ и получить
возмущение решения $\delta u_m$. Оценить число обусловленности матрицы
$\mathbf{K}$
\[
\mu_E(\mathbf{K}) \gtrsim
\frac{||\delta \mathbf{f}||_E/||\mathbf{f}||_E}
{||\delta \mathbf{u}||_E/||\mathbf{u}||_E}
\]

\problem Решить линейную систему уравнений из предыдущей задачи одним из
итерационных методов:
\begin{itemize}
\item методом Зейделя;
\item методом простой итерации с параметром $\tau =
\frac{2}{\|\mathbf{K}\|_\infty}$ (для симметричной положительно определенной
матрицы $\lambda(\mathbf{K}) < \|\mathbf{K}\|_\infty$);
\item 
\end{itemize}
Итерационный процесс следует завершить, если $||\mathbf{u}^{(k)} - \mathbf{u}^{(k-1)}|| <
\varepsilon = 10^{-6}$. В качестве начального приближения $\mathbf{u}^{(0)}$ возьмите
$\mathbf{u}^{(0)} = \mathbf{0}$.

\problem Решить следующую трехдиагональную систему уравнений методом прогонки
\[
\begin{cases}
u_0 &= 0\\
-u_{n - 1} + \big(2 + h^2\big) u_n - u_{n+1} &= 2 h^2 \sin (nh), \qquad n =
\overline{1, N-1}\\
u_N &= 0\\
\end{cases},
\]
где $N = 20, h = \frac{\pi}{N}$. Сравнить $u_n$ и $\sin(nh)$.

\problem Для задачи Коши
\[
\begin{cases}
\displaystyle\frac{dy(t)}{dt} = -2t y(t) + 2 t, \qquad t \in [0,1]\\
y(0) = 1
\end{cases}
\]
используется метод Адамса
\[
\begin{cases}
\displaystyle\frac{u_{n+1} - u_{n}}{\tau} = 3 n \tau(1 - u_n) - \tau  (n-1)(1 - u_{n-1}), \qquad n =
\overline{1, N-1}\\
u_0 = 2\\
u_1 = 2
\end{cases},
\]
где $\tau = \frac{1}{N}$.
Написать программу, реализующую этот метод, и численно убедиться, что решение
разностной задачи $u$ сходится к решению дифференциальной задачи $y(t) = 1 + e^{-t^2}$ со вторым порядком
по $\tau$.
\end{document}
