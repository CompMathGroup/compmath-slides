% !TeX program = xelatex
\documentclass[12pt]{article}

\usepackage{polyglossia}
\setmainlanguage{russian}
\setotherlanguage{english}

\setmainfont[
SmallCapsFont={Latin Modern Roman Caps},
SmallCapsFeatures={Letters=SmallCaps},
Ligatures=TeX
]{Times New Roman}

\usepackage[intlimits]{amsmath}
\usepackage{amssymb}
\usepackage{mathrsfs}
\usepackage{indentfirst}
\usepackage[colorlinks=true]{hyperref}
\usepackage[top=1cm,bottom=2cm,left=2cm,right=1.5cm]{geometry}
\usepackage{setspace}

\newcommand{\pd}[2]{\frac{\partial #1}{\partial #2}}
\newcommand{\dpd}[2]{\dfrac{\partial #1}{\partial #2}}

\newcommand{\pdd}[2]{\frac{\partial^2 #1}{\partial #2^2}}
\newcommand{\pddd}[3]{\frac{\partial^2 #1}{\partial #2\partial #3}}

\renewcommand{\arraystretch}{1.2}
\let\dividesymbol\div
\renewcommand{\div}{\operatorname{div}}
\newcommand{\grad}{\operatorname{grad}}
\newcommand{\rot}{\operatorname{rot}}
\newcommand{\const}{\operatorname{const}}
\renewcommand{\vec}[1]{\boldsymbol{\mathbf{#1}}}
\newcommand{\ten}[1]{\mathbf{#1}}
\newcommand{\cutefrac}[2]{{}^{#1}\mkern-5mu/{\!}_#2}
\newcommand{\half}{{\cutefrac{1}{2}}}
\renewcommand{\leq}{\leqslant}
\renewcommand{\geq}{\geqslant}

\newcounter{task}

\newcommand{\problem}{\par\textbf{\stepcounter{task}\thetask. }}

\title{Дополнительные задачи по курсу вычислительной математики. 5 семестр}
\author{Цыбулин Иван}

\begin{document}
\maketitle

\section{Первое задание}

\subsection{Погрешности и численное дифференцирование}

\problem Используя ваш любимый язык программирования, напишите функцию,
вычисляющую
функцию Бесселя первого рода $J_0(x)$, суммируя часть ее ряда Тейлора в
окрестности точки $x_0 = 0$:
\[
J_0(x) = \sum_{k = 0}^{\infty} \frac{(-1)^k}{(k!)^2}
\left(\frac{x}{2}\right)^{2k}.
\]

Далее, используя эту функцию и формулу численного дифференцирования,
найдите производную функции Бесселя $J_0(x)$ в точке $x = 1$ с заданной
точностью
$\varepsilon = 10^{-6}$.

Если в программе используются константы, такие как число членов ряда Тейлора или
значение шага дифференцирования, должно быть указано, как они получены.

\emph{Примечание}. Для дифференцирования использовать оптимальный шаг $h^*$.
Принять, что погрешность, с которой вычисляются значения $J_0(x)$ равна ошибке
метода вычисления функции с помощью отрезка ее ряда Тейлора, и может быть
принята равной первому отброшенному слагаемому в ряде Тейлора. Использовать
минимальное число членов ряда Тейлора для решения этой задачи. Для оценки
максимумов всех производных функции Бесселя использовать $M_n \leq 1$.

\subsection{Системы линейных алгебраических уравнений}

\problem Решить следующую трехдиагональную систему уравнений методом прогонки
\[
\begin{cases}
u_0 &= 0\\
-u_{n - 1} + \big(2 + h^2\big) u_n - u_{n+1} &= 2 h^2 \sin (nh), \qquad n =
\overline{1, N-1}\\
\hfill u_N &= 0\\
\end{cases},
\]
где $N = 20, h = \frac{\pi}{N}$. Сравнить $u_n$ и $\sin(nh)$: найти $\varepsilon = \max\limits_{0 \leq n \leq N} |u_n - \sin(nh)|$.

\problem Для решения интегрального уравнения Фредгольма второго рода
\[
u(x) - \lambda \int_{-\pi}^\pi K(x-y) u(y) = f(x)
\]
с ядром $K(s) = |s|$ и правой частью $f(x) =(1 + 2\lambda)\cos^2 \frac{x}{2} - \lambda \frac{x^2 + \pi^2}{2}$
используется квадратурная формула средней точки. При этом интегральное уравнение
сводится к следующей системе линейных уравнений
\[
\sum_{m=1}^{N} [\delta_{nm} - \lambda K_{nm}] u_{m} = f_{n}, \quad n = 1,\dots,N
\]
где $K_{nm} = h^2|n-m|$ --- симметричная матрица ядра размера $N\times N$,
$h=\frac{2\pi}{N}$, $\delta_{nm}$ --- единичная матрица, а \mbox{$f_n =
f(x_n), \; x_n = -\pi + (n-1/2)h$}. Точное решение интегрального уравнения $U(x) =
\cos^2 \frac{x}{2}$, численно должно совпадать с ним с точностью до небольшой
ошибки метода порядка $O(h^2)$. Взять $\lambda = 0.01, N = 100$.

Решить полученную систему одним из следующих методов и сравнить решение с точным
$U_m = U\left(x_m\right)$:
\begin{itemize}
\item[\textbullet${}^{\phantom{*}}$] с помощью $LU$-разложения;
\item[\textbullet${}^{*}$] с помощью $LUP$-разложения;
\item[\textbullet${}^{*}$] с помощью $LL^\mathsf{T}$-разложения (Холецкого);
\item[\textbullet${}^{*}$] с помощью $QR$-разложения методом вращений Гивенса;
\item[\textbullet${}^{*}$] с помощью $QR$-разложения методом отражений
Хаусхолдера.
\end{itemize}
Возмутить правую часть системы случайным вектором $\delta f_n$ и получить
возмущение решения $\delta u_m$. Оценить число обусловленности матрицы
системы $\mathbf{A}$
\[
\mu_E(\mathbf{A}) \gtrsim
\frac
{||\delta \mathbf{u}||_E/||\mathbf{u}||_E}
{||\delta \mathbf{f}||_E/||\mathbf{f}||_E}
\]
Оценить число обусловленности при
\[
\lambda \lesssim \lambda_\text{крит} \approx 0.07291218479495440151.
\]
\emph{Число $\lambda_\text{крит}$ является (единственным положительным) собственным значением интегрального уравнения.}

\problem Решить линейную систему уравнений из предыдущей задачи одним из
итерационных методов:
\begin{itemize}
\item методом Зейделя;
\item методом простой итерации с параметром $\tau =
\frac{2}{\|\mathbf{A}\|_\infty}$ (для симметричной положительно определенной
матрицы $\lambda(\mathbf{A}) < \|\mathbf{A}\|_\infty$);
\item SOR методом, подобрать вручную параметр релаксации $1 < \omega < 2$, при
котором сходимость будет самой быстрой.
\end{itemize}
Итерационный процесс следует завершить, если $||\mathbf{u}^{(k)} - \mathbf{u}^{(k-1)}|| <
\varepsilon = 10^{-6}$. В качестве начального приближения $\mathbf{u}^{(0)}$ возьмите
$\mathbf{u}^{(0)} = \mathbf{0}$. Сколько итераций потребовалось для сходимости?

\subsection{Переопределенные СЛАУ и МНК}

\problem Построить для функции $f(x) = \sin x$ на отрезке $[-1, 1]$
приближение многочленом степени $n = 8$ в смысле наименьших квадратов по одной из норм
\begin{itemize}
\item Многочлен наилучшего равномерного приближения
\[
\|f - g\|_C^2 = \int_{-1}^1 \frac{(f(x) - g(x))^2}{\sqrt{1-x^2}} dx.
\]
\emph{Указание}. Разложить многочлен по многочленам Чебышева
\[
P(x) = \sum_{k=0}^n c_k T_k(x), \qquad
T_k = \cos k \arccos x.
\]
и воспользоваться ортогональностью многочленов Чебышева в этой норме.
\[
\int_{-1}^1 \frac{T_n(x) T_m(x)}{\sqrt{1-x^2}} dx = \begin{cases}
\pi, & n = m = 0\\
\pi/2, & n = m \neq 0\\
0, & n \neq m,
\end{cases}
\]
\item Многочлен наилучшего приближения в норме $L_2$:
\[
\|f - g\|_2^2 = \int_{-1}^1 (f(x) - g(x))^2 dx.
\]
\emph{Указание}. Разложить многочлен по многочленам Лежандра
\begin{gather*}
P(x) = \sum_{k=0}^n c_k L_k(x), \\
L_{k+1}(x) = \frac{2k+1}{k+1} x L_k(x) - \frac{k}{k+1} L_{k-1}(x), \;
L_0(x) = 1,\; L_1(x) = x.
\end{gather*}
и воспользоваться ортогональностью многочленов Чебышева в этой норме.
\[
\int_{-1}^1 P_n(x) P_m(x) dx = \begin{cases}
\frac{2}{2k+1}, & n = m\\
0, & n \neq m,
\end{cases}
\]
\end{itemize}
Найти максимальную ошибку такого приближения на всем отрезке $[-1, 1]$.

\subsection{Нелинейные алгебраические уравнения}

\problem Решить одно из следующих уравнений
\begin{itemize}
\item Уравнение состояния реального газа Ван-дер-Ваальса
\[
\left(p + 3\frac{p_\text{кр}V_\text{кр}^2}{V^2}\right) \left(V - \frac{V_\text{кр}}{3}\right) = RT,
\]
относительно $V$ при $p = 10^5 \text{ Па}$, $T = 300\text{ К}$, $V_\text{кр} =
0.1095 \text{ м}^3 / \text{кмоль}$,
$p_\text{кр} = 3.77\cdot 10^6 \text{ Па}$, $R = 8314 \text{ Дж} / (\text{кмоль
К})$.

\item Уравнение Кеплера
\[
M = E - e \sin E
\]
относительно $E$ для $e = 0.1$ и $M = \frac{5\pi}{6}$.

\item Уравнение для зон для частицы в периодическом потенциале
\[
\cos x + \frac{a}{x} \sin x = 1
\]
относительно $x$ (найти корень с минимальным положительным $x$) при $a = 0.2$.

\item Любое нетривиальное нелинейное уравнение на выбор, например из
[\textit{Аристова Е.Н., Завьялова Н.А., Лобанов А.И.} Практические занятия по
вычислительной математике в МФТИ. Часть I, стр. 110--112.]
\end{itemize}
одним из перечисленных методов
\begin{itemize}
\item метод деления отрезка пополам;
\item метод секущих;
\item метод Ньютона.
\end{itemize}

Для каждого метода должно быть задано начальное приближение с объяснением, как
это начальное приближение было выбрано. Получить ответы с точностью не ниже
$|\Delta x| \lesssim \varepsilon = 10^{-14}$. Сколько итераций потребовалось для
сходимости?

\section{Второе задание}

\emph{В этом задании разрешается использовать библиотечные функции для решения систем линейных уравнений.}

\subsection{Системы нелинейных алгебраических уравнений}

\problem Решить одну из следующих систем алгебраических уравнений методом Ньютона:
\begin{itemize}
\item Определение положения $(x,y,z)$ и времени $t$ по четырем спутникам
\[\left\{
\begin{aligned}
(x - x_1)^2 + (y - y_1)^2 + (z - z_1)^2 = (t - t_1)^2\\
(x - x_2)^2 + (y - y_2)^2 + (z - z_2)^2 = (t - t_2)^2\\
(x - x_3)^2 + (y - y_3)^2 + (z - z_3)^2 = (t - t_3)^2\\
(x - x_4)^2 + (y - y_4)^2 + (z - z_4)^2 = (t - t_4)^2\\
\end{aligned}
\right.
\]
Координаты спутников 
$$
\mathbf r_1 = (1, 1, 1), \;
\mathbf r_2 = (-1, -1, 1), \;
\mathbf r_3 = (-1, 1, -1), \;
\mathbf r_4 = (1, -1, -1),
$$
а времена 
$$
t_1 = 2.00, \quad t_2 = 2.25, \quad t_3 = 1.70, \quad t_4 = 1.5.
$$
Найти то решение, для которого $t < t_{1,2,3,4}$.
\item Поиск орбит в одномерном отображении $\varphi(x) = r x (1 - x)$. 
\[\left\{\begin{aligned}
&x_1 = \varphi(x_n)\\
&x_2 = \varphi(x_1)\\
&\;\;\quad\vdots\\
&x_n = \varphi(x_{n-1})
\end{aligned}
\right.\]
Неизвестными являются $x_1, \dots, x_n$. Найти одно из положительных решений. Предлагается взять параметры $r = 3.5, n = 4$ или $r = 3.56, n = 8$
\end{itemize}
Самостоятельно задать начальное приближение. В качестве условия остановки метода Ньютона использовать
$$
\|\mathbf x_{n+1} - \mathbf x_{n}\|_E \leqslant \varepsilon = 10^{-14}.
$$

\subsection{Интерполяция и сплайны}

\problem Построить для функции $f(x) = \sin x$ на отрезке $[0, 2\pi]$ кубический
сплайн $P(x)$, имеющий две непрерывные производные. Численно изучить зависимость
максимального отклонения $f(x)$ и $P(x)$ в зависимости от количества узлов $N$.
Найти порядок метода (степень зависимости максимального отклонения от
$h = \frac{2\pi}{N - 1}$). В качестве концевых условий использовать точные
условия
\[
f'(0) = f'(2\pi) = 1
\]
либо
\[
f''(0) = f''(2\pi) = 0.
\]

\problem Построить для функции $f(x) = \sin x$ на отрезке $[0, 2\pi]$ кубический
сплайн $P(x)$, который в узлах интерполяции $x_k$ совпадает со значениями $f(x_k)$,
производная которого непрерывна, и в узлах совпадает со значениями $f'(x_k) =
\cos x_k$ (такой сплайн называется сплайном Эрмита). Численно изучить зависимость
максимального отклонения $f(x)$ и $P(x)$ в зависимости от количества узлов $N$.
Найти порядок метода (степень зависимости максимального отклонения от
$h = \frac{2\pi}{N - 1}$).

\subsection{Численное интегрирование}

\problem Написать программу, которая используя правило Рунге и формулу
Гаусса численного интегрирования четвертого порядка (ошибка на всем
отрезке имеет порядок $O(h^4)$)
\[
\int_{x_1}^{x_2} f(x) dx \approx \frac{x_2 - x_1}{2}\left[
f\left(\frac{x_1+x_2}{2} - \frac{x_2 - x_1}{2\sqrt{3}}\right) +
f\left(\frac{x_1+x_2}{2} + \frac{x_2 - x_1}{2\sqrt{3}}\right)
\right]
\]
вычисляет интеграл
\[
\int_0^{\pi/2} \ln \frac{1}{\sin x} dx
\]
с точностью $\varepsilon = 10^{-10}$. Сравнить полученное значение с точным
значением интеграла $\frac{\pi}{2} \ln 2$

\emph{Указание}. Убедиться, что $\Delta_h = |I_h - I_{h/2}| \sim O(h^4)$. Если это
не так, регуляризуйте подынтегральную функцию.

\problem Вычислить один из следующих интегралов
\begin{itemize}
\item Полный эллиптический интеграл первого рода
\[
K(m) = \frac{1}{2}\int_{-1}^1 \frac{dx}{\sqrt{(1-x^2)(1 - m x^2)}}
\]
\item Полный эллиптический интеграл второго рода
\[
E(m) = \frac{1}{2}\int_{-1}^1 \frac{\sqrt{1 - m x^2}}{\sqrt{1-x^2}}dx
\]
\end{itemize}
с помощью квадратурной формулы Гаусса-Чебышева для интегралов вида
\[
\int_{-1}^1 \frac{f(x)}{\sqrt{1-x^2}} dx.
\]
Изучить как точность приближения зависит от выбранного числа узлов квадратуры.
Параметр $m$ взять произвольным из интервала $0 < m < 1$. 

\emph{Примечание}. Значение для проверки можно посмотреть, например, на \href{http://www.wolframalpha.com/input/?i=K(1\%2F2)}{WolframAlpha}.

\subsection{Задача Коши для ОДУ}

\problem Для решения задачи Коши
\[
\begin{cases}
\displaystyle\frac{dy(t)}{dt} = -2t y(t) + 2 t, \qquad t \in [0,1]\\
y(0) = 1
\end{cases}
\]
использовать один из предложенных методов:
\begin{itemize}
\item метод Адамса:
\[
\begin{cases}
\displaystyle\frac{u_{n+1} - u_{n}}{\tau} = \frac{3 f(t_n, u_n) - f(t_{n-1}, u_{n-1})}{2}, & n =
1,\dots, N-1\\
u_0 = y_0\\
u_1 = y_0 + \tau f(0, y_0)
\end{cases}
\]
\item метод Эйлера с пересчетом
\[
\begin{cases}
\displaystyle\frac{\tilde u_{n+1/2} - u_{n}}{\tau/2} = f(t_n, u_n), &n =
0,\dots, N-1\\
\displaystyle\frac{u_{n+1} - u_{n}}{\tau} = 
f\left(t_n + \frac{\tau}{2}, \tilde u_{n+1/2}\right), &n =
0,\dots, N-1\\
u_0 = y_0.\\
\end{cases}
\]
В этом методе величины $\tilde u_{n+1/2}$ вспомогательные, и не являются частью сеточной функции $u$.
\end{itemize}

Написать программу, реализующую выбранный метод, и численно убедиться, что решение
разностной задачи $u$ сходится к решению дифференциальной задачи $y(t) = 1 + e^{-t^2}$ со вторым порядком.
\end{document}
