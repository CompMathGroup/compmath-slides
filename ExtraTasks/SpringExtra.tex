\documentclass[10pt]{article}
\usepackage[utf8]{inputenc}
\usepackage[T2A]{fontenc}
\usepackage[english,russian]{babel}
\usepackage{amsmath,amssymb}
\usepackage{indentfirst}

\title{Дополнительные задачи для самостоятельного решения}
\author{Цыбулин Иван}

\begin{document}
\maketitle

Данные задачи предлагаются как альтернатива соответствующим лабораторным
работам. Для реализации можно использовать любой язык программирования.

\section{Задача Коши для обыкновенных дифференциальных уравнений}
Решить одну из предложенных задач Коши
\[
\mathbf{y}'(t) = \mathbf{G}(t, \mathbf{y}), \quad \mathbf{y}(0) = \mathbf{y}_0
\]
имеющих периодические решения с периодом $T$:
\begin{enumerate}
\item[а)]
\begin{equation*}
\begin{cases}
\varphi'(t) = \omega(t)\\
\omega'(t) = -\sin \varphi(t) &\quad
0 \leq t \leq T\\
\varphi(0) = \frac{\pi}{2}, \quad \omega(0) = 0
\end{cases},
\end{equation*}
где $T = 4 K\left(\frac{\pi}{4}\right)$
\[
K(\alpha) = \int_0^1 \frac{dt}{\sqrt{(1-t^2)(1-t^2 \sin^2 \alpha)}}, \qquad
K\left(\frac{\pi}{4}\right) \approx 1.854074677301372
\]
\item[б)]
\begin{equation*}
\begin{cases}
\rho'(t) = v(t)\\
v'(t) = \rho(t)\omega^2(t) - \dfrac{1}{\rho^2(t)}\\
\varphi'(t) = \omega(t) &\quad 0 \leq t \leq T\\
\omega'(t) = -\dfrac{2\omega(t)v(t)}{\rho(t)}\\ 
\varphi(0) = 0, \quad \omega(0) = 1 / \sqrt{3},
\quad \rho(0) = 1, \quad v(0) = 0
\end{cases},
\end{equation*}
где $T = \frac{\pi}{\sqrt{2}} \left(\frac{6}{5}\right)^{3/2} \approx 2.920160646701$
\item[в)]
\begin{equation*}
\begin{cases}
u'(t) = v(t)\\
v'(t) = -u^3(t) &\quad
0 \leq t \leq T\\
u(0) = 1, \quad v(0) = 0
\end{cases},
\end{equation*}
где $T = \sqrt{32\pi} \dfrac{\Gamma(5/4)}{\Gamma(3/4)} \approx 7.41629870920549$
\end{enumerate}

В качестве численного метода используйте один из методов Рунге-Кутты выше
первого порядка, например:
\begin{itemize}
\item Метод Хойна третьего порядка

\begin{tabular}{c|ccc}
0 & 0 & 0 & 0\\
1/3 & 1/3 & 0 & 0\\
2/3 & 0 & 2/3 & 0\\
\hline
& 1/4 & 0 & 3/4
\end{tabular}
\item Метод Рунге третьего порядка

\begin{tabular}{c|cccc}
0 & 0 & 0 & 0 & 0\\
1/2 & 1/2 & 0 & 0 & 0\\
1 & 0 & 1 & 0 & 0\\
1 & 0 & 0 & 1 & 0\\
\hline
& 1/6 & 2/3 & 0 & 1/6
\end{tabular}
\item <<Классический>> метод Рунге-Кутты четвертого порядка

\begin{tabular}{c|cccc}
0 & 0 & 0 & 0 & 0\\
1/2 & 1/2 & 0 & 0 & 0\\
1/2 & 0 & 1/2 & 0 & 0\\
1 & 0 & 0 & 1 & 0\\
\hline
& 1/6 & 1/3 & 1/3 & 1/6
\end{tabular}

\end{itemize}

Проверьте, что метод имеет указанный порядок. Для этого исследуйте зависимость
отклонения за период численного решения 
$\varepsilon = \|\mathbf{y}(T) - \mathbf{y}(0)\|$ от величины шага.

\section{Жесткая задача Коши для ОДУ}

Найти численное решение одной из следующих жестких задач
\begin{itemize}
\item[а)] <<Брюсселятор>>
\begin{gather*}
\frac{dx(t)}{dt} = A + x^2(t) y(t) - (B + 1) x(t),\\
\frac{dy(t)}{dt} = B x(t) - x^2(t) y(t),\\
x(0) = 2, y(0) = 3, A = 1, B = 3, \qquad 0 \leq t \leq 20
\end{gather*}
\item[б)]
\begin{gather*}
y'(t) = 50 (\cos t - y^3(t)), \qquad 0 \leq t \leq 20\\
y(0) = 0
\end{gather*}
\end{itemize}

В качестве численного метода используйте один из следующих неявных методов:
\begin{itemize}
\item Неявный метод Эйлера первого порядка
\begin{gather*}
\begin{cases}
\dfrac{\mathbf{u}_{n+1} - \mathbf{u}_{n}}{\tau} = \mathbf{G}(t_{n+1},
\mathbf{u}_{n+1})\\
\mathbf{u}_0 = \mathbf{y}(0)
\end{cases}
\end{gather*}
\item Диагонально-неявный метод Рунге-Кутты с таблицей Бутчера
\begin{center}
\begin{tabular}{c|cc}
$1 - \frac{\sqrt{2}}{2}$ & $1 - \frac{\sqrt{2}}{2}$ & 0\\
$1$ & $\frac{\sqrt{2}}{2}$ & $1 - \frac{\sqrt{2}}{2}$ \\
\hline
& $\frac{\sqrt{2}}{2}$ & $1 - \frac{\sqrt{2}}{2}$
\end{tabular}
\end{center}
\item ФДН\footnote{Формула дифференцирования назад}-метод второго порядка
\begin{gather*}
\begin{cases}
\dfrac{3\mathbf{u}_{n+1} - 4\mathbf{u}_{n} + \mathbf{u}_{n-1}}{2\tau} = \mathbf{G}(t_{n+1},
\mathbf{u}_{n+1})\\
\mathbf{u}_0 = \mathbf{y}(0)\\
\mathbf{u}_1 = \mathbf{u}_0 + \tau \mathbf{G}(\tau, \mathbf{u}_1)
\end{cases}
\end{gather*}
\end{itemize}

Для решения нелинейных уравнений используйте метод Ньютона.

Определить максимальный шаг по времени, при котором \emph{явный} метод Эйлера
для данной задачи Коши устойчив. Убедиться, что используемый метод устойчив при
гораздо больших шагах по времени.

\section{Нелинейная краевая задача}
Форму волн на поверхности воды в некотором приближении можно описать 
уравнением
\begin{gather*}
\phi''(x) + 3 \phi^2(x) - \phi(x) = 0
\end{gather*}

Найдите методом линеаризации решение, удовлетворяющее условиям
\begin{gather*}
\phi(0) = 0.027923117357796894 \qquad
\phi(5) = 0.498518924361681638
\end{gather*}

Проверить, что решение дополнительно удовлетворяет условиям \[\phi'(0) =
\phi'(5) = 0.\]
Последнее означает, что решение можно периодически продолжить на всю числовую
ось.

\section{Спектральная задача для оператора Штурма-Лиувилля}
Найти волновую функцию $\psi(x)$ и уровень энергии $E$ основного состояния в одной из следующих
задач для уравнения Шрёдингера:
\begin{itemize}
\item Квантовый гармонический осциллятор
\begin{gather*}
-\frac{1}{2}\psi''(x) + \frac{x^2}{2} \psi(x) = E\psi(x)
\end{gather*}
Волновая функция основного состояния удовлетворяет условию $\psi'(0) = 0$, а
также условию асимптотики на бесконечности
\[
\lim_{x\rightarrow +\infty} \psi'(x) + \sqrt{x^2 - 2E} \psi(x) = 0,
\]
которое для простоты можно перенести в конечную точку $x = L \equiv 5$
\begin{gather*}
-\frac{1}{2}\psi''(x) + \frac{x^2}{2} \psi(x) = E\psi(x)\\
\psi'(0) = 0\qquad
\psi(L) + \sqrt{L^2 - 2E} \psi(L) = 0
\end{gather*}

Найти нетривиальное решение с минимальным $E > 0$.
\item Треугольная потенциальная яма
\[
U(x) = 
\begin{cases}
+\infty, &x < 0\\
x, &x \geq 0
\end{cases}
\]
Уравнение Шрёдингера принимает вид
\begin{gather*}
-\frac{1}{2}\psi''(x) + x \psi(x) = E\psi(x)
\end{gather*}
Поскольку при $x < 0$ волновая функция равна нулю, $\psi(0) = 0$.
Также должно выполняться условие асимптотики на бесконечности
\[
\lim_{x\rightarrow +\infty} \psi'(x) + \sqrt{2x - 2E} \psi(x) = 0,
\]
которое для простоты можно перенести в конечную точку $x = L \equiv 5$
\begin{gather*}
-\frac{1}{2}\psi''(x) + x \psi(x) = E\psi(x)\\
\psi(0) = 0\qquad
\psi(L) + \sqrt{2L - 2E} \psi(L) = 0
\end{gather*}

Найти нетривиальное решение с минимальным $E > 0$.

\end{itemize}

\newcommand{\pd}[2]{\frac{\partial #1}{\partial #2}}
\section{Уравнение переноса}

Решить уравнение переноса
\begin{gather*}
\pd{u}{t} + \pd{u}{x} = 0, \qquad 0 \leqslant x \leqslant 1, \quad 0 \leqslant t
\leqslant 0.5\\
u\big|_{t = 0} = 
\begin{cases}
0, &x > 0.2\\
1, &x \leqslant 0.2
\end{cases}\\
u\big|_{x = 0} = 1,
\end{gather*}
используя следующую разностную схему:
\begin{gather*}
u_0^{n+1} = u_1^{n+1} = 1,\\
u_{m}^{n+1} = (1 - \sigma)u_m^n + 
\sigma u_{m-1}^n + \frac{\sigma (\sigma - 1)}{2}
\left(
	f(u_{m-1}^n, u_{m}^n, u_{m+1}^n) -
	f(u_{m-2}^n, u_{m-1}^n, u_{m}^n)
\right),\\
\frac{u_M^{n+1} - u_M^n}{\tau} + 
\frac{u_M^n - u_{M-1}^n}{h} = 0.
\end{gather*}
Пусть $M = 100, h = \dfrac{1}{M}, \tau = \dfrac{h}{2}, \sigma = \dfrac{\tau}{h}
= \frac{1}{2}$.

\smallskip
Использовать одну из следующих функций $f(x, y, z)$:
\begin{itemize}
\item Схема Лакса-Вендроффа
\[
f(x, y, z) = z - y
\]
\item Схема Бима-Уорминга
\[
f(x, y, z) = y - x
\]
\item Схема Фромма
\[
f(x, y, z) = \frac{z - x}{2}
\]
\item TVD схема с ограничителем minmod
\[
f(x, y, z) = \operatorname{minmod}(z - y, y - x)
\]
\item TVD схема с ограничителем superbee
\[
f(x, y, z) = \operatorname{maxmod}
\Big[\operatorname{minmod}\big(2(z - y), y -
x\big),
\operatorname{minmod}\big(z - y, 2 (y - x)\big)
\Big]
\]
\item TVD схема с ограничителем ван Альбада
\[
f(x, y, z) = \frac{(z-x) \max(0, (z-y)(y-x))}{(z-y)^2 + (y-x)^2 + 10^{-14}}
\]
\item TVD схема с ограничителем ван Лира
\[
f(x, y, z) = \frac{|z-y|(y-x) + |y-x|(z-y)}{|z-y| + |y-x| + 10^{-14}}
\]
\end{itemize}
Здесь
\begin{gather*}
\operatorname{minmod}(a,b) = 
\begin{cases}
0, &ab\leqslant 0\\\min(a,b), &a>0, b>0\\\max(a,b), &a<0, b<0
\end{cases}\\
\operatorname{maxmod}(a,b) = 
\begin{cases}
0, &ab\leqslant 0\\\max(a,b), &a>0, b>0\\\min(a,b), &a<0, b<0
\end{cases}
\end{gather*}

Провести сравнение решения с решением по схеме с $f(x,y,z) \equiv 0$ (левый уголок).

\section{Параболическое уравнение}
Решить нестационарное уравнение Шрёдинрега
\[
\pd{\Psi}{t} = \frac{i}{200} \pd{{}^2\Psi}{x^2}, \qquad 0 \leqslant x \leqslant
1,\quad 0 \leqslant t \leqslant 1.3, \quad i = \sqrt{-1}
\]
с начальным условием
\[
\Psi\big|_{t = 0} = \exp\left\{100ix - \frac{(x-0.3)^2}{0.01}\right\}
\]
и граничными условиями
\[
\Psi\big|_{x=0} = \Psi\big|_{x=1} = 0.
\]
Использовать одну из следующих схем:
\begin{itemize}
\item Схема Кранка-Никольсон
\begin{gather*}
\frac{\Psi^{n+1}_m - \Psi^{n}_m}{\tau} = \frac{i}{400}
\left[
\frac{\Psi^{n+1}_{m-1} -2\Psi^{n+1}_m + \Psi^{n+1}_{m+1}}{h^2} + 
\frac{\Psi^n_{m-1} -2\Psi^n_m + \Psi^n_{m+1}}{h^2}\right]\\
\Psi^{n+1}_0 = \Psi^{n+1}_M = 0
\end{gather*}
\item Неявная схема
\begin{gather*}
\frac{\Psi^{n+1}_m - \Psi^{n}_m}{\tau} = \frac{i}{200}
\frac{\Psi^{n+1}_{m-1} -2\Psi^{n+1}_m + \Psi^{n+1}_{m+1}}{h^2}\\
\Psi^{n+1}_0 = \Psi^{n+1}_M = 0
\end{gather*}
\end{itemize}
Принять $M = 1000, h = \dfrac{1}{M}, \tau = h$

\end{document}
